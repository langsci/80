\chapter{The problem of dual aspectual forms}\label{ch:3}\label{sec:3}
\section{Introduction}\label{sec:3.0}

In this chapter, I will outline how authors have attempted to address
what they evaluate as the problem presented by some property items,
namely those which appear in multiple uses and specifically those that
I refer to as \textsc{dual aspectual forms}.\footnote{Recall that my use of the
  term ``dual aspectuality'' is intended to capture the fact that these
  items appear in both \isi{Stative} and Non-stative use.}  Typical items
as \textit{ded, weeri, sik}, etc., may be used to express the \isi{Stative}
meanings `dead’ `be tired', and `be sick' respectively as well as the
Non-stative meanings `die',  `become\slash make tired' and `become\slash make
sick'. As indicated previously (\chapref{ch:1}) the two issues
that have been addressed regarding this group of items are:

\begin{enumerate}
\item Their \isi{categorial status} as either verbs or adjectives
\item Their \isi{aspectual status} in relation to the \isi{Stative}\slash Non-stative
  distinction
\end{enumerate}

In this work, both issues will be treated as logically related to each
other, with the issue of the \isi{aspectual status} of these items being
primary in relation to the question of \isi{categorial status}.  From this
perspective, the determination of the \isi{aspectual status} of a lexical
item in this group as either \isi{Stative} or Non-stative is logically tied
to a status as adjective or verb respectively.  Thus there is
basically one major question to be answered and the other follows as
a consequence.  This however is not necessarily the way in which these
items have been treated in the literature.\footnote{Note for example
  that while \citet{Kouwenberg1996} assumes adjectival status for the
  \isi{Stative} use, and verbal status for Non-stative use, in the case of
  \citet{Winford1993}, there is a division on the basis of \isi{Stativity}
  but he assumes a group consisting solely of verbs.} Thus, I will
outline the treatment of these as two issues in the sections which
follow.

In \sectref{sec:3.1} I will evaluate the way in which various authors
have attempted to deal with the issue of the \isi{categorial status} of this
group of items.  The discussion will focus, first, on the debate
between authors \citet{Sebba1986} and \citet{Seuren1986}, who take
seemingly opposed positions on the matter: Sebba adopts the ``standard”
view that these items are (\isi{Stative}) verbs; Seuren adopts
the view that they are adjectives introduced by a null copula. We will
see that neither of these viewpoints can deal satisfactorily with the
diversity that is presented in the data, and that a more
differentiated approach is called for.  Finally, I will consider the
arguments put forward by \citet{Kouwenberg1996}, who analyses this
group of items as containing both verbs and adjectives.

In \sectref{sec:3.2}, I will look at the question of the aspectual
status of \isi{dual aspectual forms} and the \isi{Stative}\slash Non-stative
distinction.  This question of the \isi{aspectual status} of these items is
less overtly discussed than that of their \isi{categorial status}, but
\citet{Winford1993} represents the most complete attempt to address
this group of items from this perspective.  His treatment will be
outlined in \sectref{sec:3.3} and evaluated based on some JC\il{Jamaican Creole} data.  An examination of Winford’s proposal will point to the desirability of an
alternative model which categorises items based on their aspectual
behaviour as determined through a combination of syntactic and
semantic criteria, rather than the semantic notions of
\citet{Dixon1977} which Winford draws on.  Such a model will be
presented in \chapref{ch:5}.  I will return to the issue of categorial
status in \chapref{ch:6}.

\section{Verb or adjective? The categorial status of property
  items}\label{sec:3.1}


\subsection{The Sranan case: A debate between Sebba and
  Seuren}\label{sec:3.1.1}

As indicated in \chapref{sec:2}, the standard analysis for property
items in CECs since \citet{Voorhoeve1957} is that these items in
\isi{predicative use} are essentially (\isi{Stative}) verbs
(cf. \citealt{Alleyne1980,Jaganauth1987}; etc).  This is the view
adopted by \citet{Sebba1986}.  Nonetheless, based on data from \ili{Sranan}
(SR\il{Sranan}), \citet{Sebba1986} argues that a form such as \textit{bradi}
`broad' has dual \isi{categorial status} appearing as both an adjective
(\ref{ex:3:1}a) and a (\isi{Stative}) verb (\ref{ex:3:1}b).  This
classification is based on its syntactic appearance with or without
the locative\slash existential copula \textit{de} and modifying material
such as `so'.  Compare:

\ea%1
\label{ex:3:1}
SR\il{Sranan} (adapted from \citealt[112]{Sebba1986})\\
\ea
\gll A liba     de {\ob} so \textbf{bradi} {\cb}.\\
    art river {\COP} {}  so         broad {}\\
\glt `The river is so broad.'

\ex
\gll A liba {\ob} \textbf{bradi} so {\cb}.\\
	art river        {\ob} broad so ]\\
\glt `The river is so broad.'' \z \z

Thus, in (\ref{ex:3:1}a) \textit{bradi} satisfies the syntactic
condition of an adjective by appearing with the copula \textit{de}
while it is said to be a \isi{Stative} verb where it appears without the
copula in (\ref{ex:3:1}b).  I will return to the significance of
(\ref{ex:3:1}a) below.  Based on the \isi{predicative use} of an item such
as \textit{bradi} in (\ref{ex:3:1}b), \citet{Sebba1986} argues that

\begin{quote}
while there is a separate category ``adjective” in \ili{Sranan} , the class
of objects which corresponds to ``\isi{predicate adjective}” in English must
be regarded as verbs in \ili{Sranan}. (p.~110)
\end{quote}

It is worth noting that Sebba seems to address here the entire class
of predicates labeled ``predicate adjectives'' in English.  This would
essentially coincide with the class of property items that I discuss
below in \sectref{sec:3.3} (see \tabref{extab:3:17}, ``Property concepts
in CECs'').  As we will see there, these constitute a diverse group of
items based not only on their semantic denotations but also on their
syntactic behaviour.  Sebba’s arguments for predicative property items
as \isi{Stative} verbs, however, are based on differences and similarities
that he observes between what he labels as attributive adjectives,
predicative adjectives and \isi{Stative} verbs.

Sebba treats adjectives as a subclass of \isi{Stative} verbs based on what
he considers to be the ``well-known semantic similarity between verbs
and predicate adjectives” and the ``obvious similarity in the syntactic
behavior of \ili{Sranan} stative verbs like \textit{lobi} `like, love' and
predicate adjectives like \textit{tranga} `strong'” (p.~114).  The
examples in \xxref{ex:3:2}{ex:3:5} below show their distributional similarity (all
adapted from \citealt[114]{Sebba1986}):

\ea%2
\label{ex:3:2}
\ea
\gll Rudy \textbf{lobi} dagu so.\\
     Rudy         love dog so\\
\glt `Rudy so likes dogs.'

\ex
\gll Rudy \textbf{tranga} so.\\
     Rudy         strong  so\\
\glt `Rudy is so strong.'\z \z

\ea%3
\label{ex:3:3}
\ea
\gll Rudy ben \textbf{lobi} dagu.\\
    Rudy {\TNS} love dog \\
\glt `Rudy loved dogs.'

\ex
\gll Rudy ben \textbf{tranga} so.\\
     Rudy {\TNS} strong so\\
\glt `Rudy was so strong.'\z \z

\ea%4
\label{ex:3:4}
\ea
\gll Rudy              e \textbf{lobi} dagu.\\
          Rudy \textsc{asp}      love dog\\
\glt `Rudy is starting to like dogs.'

\ex
\gll Rudy                  e \textbf{tranga}. \\
        Rudy \textsc{asp} strong\\
\glt `Rudy is getting strong.'\z \z

As these examples indicate, the distribution of a form such as
\textit{tranga} `strong' as it relates to Tense-\isi{Aspect} markers
appears to be identical to that of a \isi{Stative} verb such as
\textit{lobi} `like, love'.  Moreover, \REF{ex:3:5} shows that neither
\textit{tranga} nor \textit{lobi} is acceptable after a copula:

\ea%5
\label{ex:3:5}

\ea[*]{
\gll  Rudy de \textbf{lobi} dagu\\
      Rudy {\COP}      love dog\\}
\ex[*]{
\gll Rudy de \textbf{tranga}\\
     Rudy {\COP} strong\\}
\z \z

Recall (\ref{ex:3:1}a), which seemed to show that it is possible for a
predicative adjective to be introduced by a copula.  Sebba points out
that it is the presence of \textit{so} or other types of quantifying elements
which makes this possible; he calls the phrases that are thus formed
``Extent Phrases” and claims that these provide a context for the
\isi{predicative use} of adjectives.

Sebba’s analysis centers as well on the obvious differences in the
distribution of attributive and predicative adjectives.  In this
regard he notes that

\begin{quote}
attributive adjectives precede the nominal they modify; any
quantifiers, etc. which modify them occur before the adjectives.  This
is in contrast to the behaviour of modifiers with \textit{predicate}
adjectives which in most cases \textit{follow} the adjective. (p.~114--115)\end{quote}

This is based on observation of data such as that shown below in
\REF{ex:3:6} and \REF{ex:3:7}:

\ea%6
\label{ex:3:6}
Attributive adjectives \citep[115]{Sebba1986}\\
 \ea
 \gll  {wan} \textbf{bigi}  {dagu}\\
       a big dog\\

  \ex 
  $\left\{ 
	  \begin{tabular}{ll}
	  \exfont (wan) & \exfont (tumsi)\\
	  {\db}a & {\db}too \\
	  \\
	  \exfont (someni) &\exfont (moro)\\
	  {\db}so.many & {\db}more \\
	  \end{tabular}
  \right\}$
  \gll \textbf{bigi} dagi\\
  big dog\\
  \z
\z


\ea%7
\label{ex:3:7}
Predicative adjectives \citep[115]{Sebba1986}\\
\ea
\gll A dagu bigi tumsi.\\
     the dog big too.much\slash very\\
\glt `The dog is too big.'

\exbox{
$\left.\text{\parbox{3.25cm}{
      \gll  A   dagu moro bigi\\
	        the dog  more big\\   
      \ex
      \gll A dagu bigi moro\\
           the dog big more\\  
}}\right\}$ \gll leki trawan. \\
                than the.other.one \\
\glt `The dog is bigger than the other one.'
}
\z
\z

This difference in the distribution of
predicative and attributive adjectives, leads Sebba to conclude that,
``in \ili{Sranan}, attributive adjectives must be treated as a class distinct
from both verbs and predicate adjectives.” (p.~115). According to him,

\begin{quote}
there is reason to recognize an independent category \isi{Adjective} […] in
\ili{Sranan}, but […] predicate adjectives are in fact members of the
category V (verb) and behave like stative verbs. (p.~116)
\end{quote}

\citet{Seuren1986} provides a different analysis of predicative
adjectives, which he treats as adjectives in \ili{Sranan} regardless of
their syntactic realizations. Using the similar case of the presence
or absence of the copula \textit{de,} with such items as a part of his
evaluation, (cf. \ref{ex:3:1} above) he acknowledges the existence of
cases where the presence of \textit{de} signals a difference in
meaning as in \REF{ex:3:8} below:

\ea%8
\label{ex:3:8}
\ea
\gll A \textbf{bun}.\\
\textsc{3sg} good\\
\glt `That\slash he is ok.'
\ex
	\gll A \textbf{de} \textbf{bun}.\\
\textsc{3sg} {\COP} good\\
\glt `He is doing alright.'\\
\z \z

The semantic difference signaled here based on the translation
provided, is one between an individual and stage level interpretation,
(\ref{ex:3:8}a) and (\ref{ex:3:8}b) respectively.  However, according
to Seuren, variations in meanings due to the presence of \textit{de}
are ``few in number and not of a general nature ” (p.~124). More
frequent it seems are ``regular and predictable alternations” such as
those in \REF{ex:3:9} and \REF{ex:3:10} below:

\ea \label{ex:3:9}
\ili{Sranan} (adapted from \citealt[124]{Seuren1986})\\
\ea
\gll  Mi futu  no \textbf{bigi} \textbf{so}.\\
      \textsc{1sg} foot  \textsc{neg} big so\\
\glt `My foot is not so big.'\\

\ex
\gll Mi futu  no      de   so \textbf{bigi}.\\
         \textsc{1sg}  foot  \textsc{neg} {\COP} so big\\
\glt `My foot is not so big.'\\
\z \z


\ea\label{ex:3:10} \ili{Sranan} (adapted from \citealt[124]{Seuren1986}) 
\ea
\gll  A liba \textbf{bradi}.\\
\textsc{art} river broad\\
\glt    `The river is wide.'

\ex
\gll  O bradi a liba \textbf{bradi}?  \\
    how broad \textsc{art} river broad\\
\glt `How wide is the river?'

\ex
\gll  O \textbf{bradi} a liba de?    \\
     how broad \textsc{art} river {\COP}\\
\glt `How wide is the river?'

\ex
\gll A liba musu \textbf{bradi}.\\
\textsc{art} river must broad\\
\glt `The river must be wide.'

\ex
\gll  A liba musu \textbf{de} \textbf{bradi}.\\
\textsc{art} river must {\COP} broad\\
\glt `The river must be wide.'\z \z

As noted here, the copula \textit{de} with an item such as
\textit{bradi} varies in its syntactic appearance without any notable
change in meaning.  Based on data such as this, Seuren posits that the
presence or absence of \textit{de} is predictably linked to the
existence of an underlying copula \textsc{be}.  According to him, this \textsc{be}
copula,

\begin{quote}
manifests itself as a zero morpheme (ø) when it finds itself in the
position of a finite verb form and is followed directly by an
adjective, but as \textit{na} or \textit{de} otherwise, in the same
position. When \textsc{be} is infinitival, the use of \textit{de} is optional
when it is directly followed by an adjective; otherwise it is
obligatory.  (p.~124)
\end{quote}

This observation essentially allows him to treat the various
occurrences of items such as \textit{bigi} `big' and \textit{bradi}
`broad' simply as adjectives where Sebba posits a difference in
\isi{categorial status} in the different instantiations.

From Seuren’s perspective the standard analysis of predicative
adjectives as (\isi{Stative}) verbs ``seems to have a lot going for it, since
on superficial inspection, adjectives seem to behave like verbs”.
However, he points out that the ``parallelism breaks down when the
facts are inspected more closely'' (p.~123).  In particular, he
highlights a difference between \isi{Stative} verbs and adjectives whereby
``adjectives but not stative verbs, allow for \textit{causative} uses
as well” (p.~127).  For example, an item such as \textit{tranga}
 `strong’ may be used as a causative while the same is not possible for
a verb like \textit{lobi} `love'.  Compare:

\ea%11
\label{ex:3:11}
\ili{Sranan} \citep[127]{Seuren1986}\\
\ea[]{
\gll Alen e tranga yu.\\
    rain \textsc{asp} strong you      \\
\glt `Rain makes you strong.'}

\ex[*]{
\gll Sopi e lobi yu a uma dati.\\
		booze makes you love the woman that  \\
\glt `Booze makes you love that woman.'} \z \z


As shown here (in \ref{ex:3:11}), a \isi{causative variation} is possible with an item such as
\textit{tranga} ‘strong’ but not with a regular \isi{Stative} verb like
\textit{lobi} `love'.  This and other differences in the
distributional properties of \textit{tranga} and \textit{lobi} show,
according to Seuren, ``that there \textit{is} a difference between
predicate adjectives and stative verbs” \citep[127]{Seuren1986}.

A general objection to both Sebba’s and Seuren’s positions is that
neither is able to deal with variation in the class of items they
consider.  It is interesting for me that both Sebba and Seuren use
individual items, such as \textit{bradi} `broad' or \textit{tranga} `strong' 
to represent the distribution of predicative property
items in \ili{Sranan}.  A question that logically arises here for me, is
whether or not the distribution of an adjective such as \textit{tranga}
`strong' is the same as that of, for instance, \textit{bigi} `big'
or \textit{bradi} `broad'.  Below, in \sectref{sec:3.3}, we will see
that property items are varied in their behaviour.  This means that
the conclusion drawn by Sebba of a clear similarity in the
distribution of predicate adjectives and \isi{Stative} verbs on the basis of
the behaviour of a single item is invalidated.  At best, it may only
be applicable to a particular class of predicative property items
rather than to this class in general.  Similarly, not every property
item is able to participate in the \isi{causative variation} which Seuren
points to as a property which distinguishes these items from stative
verbs.  In short, an analysis which generalises over the behaviour of
predicative adjectives must consider more closely the behaviour of the
range of property items rather than just a few individual
items.\footnote{Another weakness in Sebba’s argumentation can be seen
  in the fact that he, despite his view that predicative property
  items are \isi{Stative} verbs, nonetheless needs these items to be
  distinct from \isi{Stative} verbs on the basis of their
  multi-functionality. He points out that ``all \ili{Sranan} adjectives may
  also function as nominals which denote their abstract quality,
  e.g. \textit{ogri}, Adj: `ugly, bad' ; N: `evil (deed)';
  \textit{fri} Adj.: `free' N: `freedom'. This possibility of
  multi-functionality applies across the board to \ili{Sranan} adjectives
  but only to a subset of verbs. e.g., \textit{singi} V: `sing' N:
  `song''. Thus while multifunctional verbs would either have to be
  listed as both V and N in the lexicon [...] the label adjective (A)
  would be sufficient to mark an item as also a member of the category
  N” (p.~116)}

This point can be illustrated by considering the variation in
transitivity (and \isi{Stativity}) that items such as \textit{tranga}
`strong\slash make strong' display (see example \ref{ex:3:11}a), or even the simple
fact that every one of these items is at least \textit{some} times an
adjective.  Both Seuren and Sebba attempt to account for the syntactic
appearance of the property items: Sebba, by positing two categories
for these items based on their syntactic realisation (attributive
vs. predicative) and Seuren by positing something of an abstract
generative device that predicts the different appearances, linking
these to one category (adjective).  But neither of these treatments
accounts for the fact that these items are able to behave the way they
do in the first place -- an issue which I will attempt to address in
Chapters~\ref{ch:4} and~\ref{ch:5}.\largerpage[1.5]

In \sectref{sec:5.2}, I will posit for cases similar to
\textit{tranga} `strong\slash make strong' that Non-stative elements of
{\textup{meaning}} such as \CAUSE or \BECOME are present or may be
introduced in the Event Structure of such items allowing for this
variation.  It is the presence of these at the lexico-semantic level
that distinguishes inherently \isi{dual aspectual forms} (Transitions) from
regular \isi{Stative} verbs which have an Event Structure of \isi{State}.  This
means that predicate adjectives of the type indicated by
\textit{tranga} `strong' are indeed semantically distinct from \isi{Stative}
verbs; however, they are also distinct from other predicate adjectives
which do not allow for this \isi{causative variation}\footnote{While I do
  not deal with this issue here, indications are that there may be
  some kind of semantic feature associated with these items that sets
  them apart in the lexicon as vulnerable to the causative
  variation. Based on \citet{Winford1993}, this may be the strength of
  a feature akin to the notions \textsc{transitory} or \textsc{permanent} whereby items
  which are most transitory would be those most likely to be
  affected. This of course would vary according to speech community
  accounting for the differences in the behaviour of similar items
  across CECs.} – a possibility not considered by Seuren.

In the section which follows, I will look at the analysis presented by
\citet{Kouwenberg1996} which may be said to be distinct from that of
both \citet{Sebba1986} and \citet{Seuren1986} in its clear acknowledgement
of the categorial diversity of this group of items.

\subsection{\citet{Kouwenberg1996}}\label{sec:3.1.2}

A defining feature of \citet{Kouwenberg1996} is her recognition of a
diverse \isi{categorial status} for the group of adjectival predicates\footnote{Note
  that Kouwenberg utilises the term ``adjectivals'' in reference to this
  group of items.}  in Caribbean Creoles.  According to
her, ``part of the problematic nature of the issue results from
attempts to treat a large and diverse class of forms as a single class
for which a unified account is sought” (p.~27).  Thus, in contrast to
other authors who argue for the status of adjectival predicates as
either verbs or adjectives, \citet{Kouwenberg1996} acknowledges a
group of forms comprising both (Non-stative) verbs and adjectives.
With reference to data from \ili{Saramaccan} (SM\il{Saramaccan}), \ili{Sranan} (SR\il{Sranan}) and 
\ili{Berbice Dutch Creole} (BD\il{Berbice Dutch Creole}), she argues for the existence of a class of
adjectives which have related verbs.  According to her:

\begin{quote}
In view of the existence of a class of forms that may appear in the
\isi{attributive position}, as complements of copular verbs, in comparative
constructions, and in -- SR\il{Sranan} -- in question phrases, the existence of a
class of adjectives in the Creole languages under discussion is, I
think, indisputable.  That these languages also have verbs which are
somehow related to these adjectives follows from facts such as the
ability of these forms to appear as predicates with an imperfective
marker, to participate in predicate cleft, and to take object NPs.
(p.~32)
\end{quote}

From this perspective, she does not argue either against a position
that posits adjectivals as verbs or as adjectives but rather against
the idea that these items fall into one single class.

In examples such as \REF{ex:3:12}, she points out that the attributive
elements \textit{satu} `salted' (SM\il{Saramaccan}), \textit{bradi} `broad' (SR\il{Sranan}) and
\textit{potɛ} `old' (BD\il{Berbice Dutch Creole}) are adjectives:

\ea%12
\label{ex:3:12}
\ea
\ili{Saramaccan} \\
\gll        di satu gwamba \\
	\textsc{art} salt meat  \\
\glt     `the salted meat'

\ex
\ili{Sranan}\\
\gll        a bradi liba \\
	\textsc{art} broad river    \\
\glt      `the wide river'

\ex
Berbice Dutch\il{Berbice Dutch Creole}\\
\gll        di potɛ jɛrma   \\
	\textsc{art} old woman  \\
\glt `the old woman''  \z \z

The behaviour of adjectivals in this position seems quite uniform and
as Kouwenberg points out distinguishes them from ``real verbs'' which
for the most part do not appear in such positions.\footnote{\citet{Winford1993} points to a small group of   verbs in JC\il{Jamaican Creole} and GC\il{Guyanese Creole} which appear in \isi{attributive position}.} (p.~30).

Where such forms are similar to verbs in terms of their ability to
appear with \isi{Imperfective} aspect, Kouwenberg highlights a difference in
the behaviour of some items as opposed to others.  In this regard, she
points out that ``there is a class of BD\il{Berbice Dutch Creole} adjectivals which pattern
fully with action verbs such as \textit{kain} `pick' in that they
appear quite unproblematically with aspectual suffixes” while there
are those which ``may appear in perfective forms but not in
imperfective forms” (p.~30).  This difference is highlighted in the
behaviour of BD\il{Berbice Dutch Creole} items such as \textit{gu} `big', which patterns with
Non-stative verbs, as opposed to \textit{potɛ} `old', which does not
show the same range of possibilities, as shown in \REF{ex:3:13}:\largerpage[2]

\ea%13
\label{ex:3:13}
\citep[30]{Kouwenberg1996} \ea
Berbice Dutch (BD\il{Berbice Dutch Creole})\\
\gll  Titi      ju \textbf{gwarɛ}...       \\
         time \textsc{2sg} big.{\IPF}        \\
\glt `When you are growing up...'

cf. \gll  ju krikja gu\\
\textsc{2sg} get.{\IPF} big\\
\glt `you are getting big\slash growing up.'


\ex
Berbice Dutch\il{Berbice Dutch Creole}\\
\gll         Eni masi gugutɛ nau.          \\
	\textsc{3pl} must big.big.{\PF} now                   \\
\glt       `They must be big\slash have grown up by now.'

cf.
\gll        Eni gu.\\
	\textsc{3pl} big \\
\glt      `They are big.' (inherent or acquired) \z \z

\ea%14
\label{ex:3:14}

Berbice Dutch\il{Berbice Dutch Creole}\\
\ea[]{
\gll (X)   potɛtɛ na, timi kori ababaka.     \\
(X) old.{\PF} now, able work anymore.\textsc{neg}  \\
\glt `(X) has got old, (he) cannot work anymore.'\\
cf.
\gll O pot{ɛ}.\\
           \textsc{3sg} old\\
\glt `He is old.'}

\ex[*]{ o pota \\
cf. \gll O krikja potɛ.\\
% 	\textsc{3sg} old-{\IPF}\\
	\textsc{3sg} get.{\IPF} old\\}
\glt `She is getting old.''  \z \z %\todo{removed \textsc{3sg} old-{\IPF}}

In these examples, Kouwenberg points out that while, \textit{gu} `big'
can appear in both \isi{Perfective} and \isi{Imperfective} use, \textit{potɛ}
 `old' is more restrictive in that while it appears in \isi{Perfective} use,
``a process interpretation can be expressed only by use of a
construction which contains a copular verb.” (p.~30)

Due to the difference noted in the behaviour of \textit{gu} `big' on
one hand and \textit{potɛ} `old' on the other, Kouwenberg assumes two
classes of adjectivals. Essentially for her, \textit{gu} `big'
``belongs simultaneously to the class of adjectives and the class of
(intransitive process) verbs”, while, \textit{potɛ} `old', “joins
the class of verbs through a productive derivation which relates
derived intransitive process verbs to base adjectives.” (p.~36).  In
recognising two groups of adjectivals, \citet{Kouwenberg1996} is most
similar to \citet{Winford1993} who also notes a split in this group of
items based on their compatibility with \isi{Imperfective} aspect.  However,
where Kouwenberg distinguishes two groups, comprising verbs and
adjectives, Winford sees a group of verbs differentiated based on
\isi{Stativity}: one group is \isi{Stative} while the other is Non-stative).

\subsection{ A note on later works}\label{sec:3.1.3}

\citet{Winford1997} and \citet{Migge2000} are among later authors to
weigh in on the discussion of the \isi{categorial status} of these items.
Similar to \citet{Kouwenberg1996} these authors recognise a flexible
categoriality associated with property items but align more closely
with the analysis of these items as verbs as proffered by
\citet{Sebba1986}.  Specifically, both \citet{Winford1997} and
\citet{Migge2000} uphold treatment of these items as verbs displaying
flexible \isi{Stativity}, but also make allowances for these items as
``adjectives in certain functions” \citep[249]{Winford1997} when
occurring as prenominal attributives.  In treating these items,
Winford (1997) calls on syntactic criteria such as their ability to appear
with TMA markers and adverbial modifiers; their ability to undergo
``predicate cleft” and their appearance in modifying serial verb
constructions (p.~257).  \citet{Migge2000} reflects a similar
analysis.  Neither of these works contributes new arguments to the
discussion on the \isi{categorial status} of these items.

In the case of Winford, while he addresses the \isi{categorial status} of
these items, his treatment of their \isi{aspectual status} as put forward in
his (\citeyear{Winford1993}) work reflects, in my estimation, his principal
contribution to this discussion and this is the view that underlies
the analysis that we see in later works.  In \sectref{sec:3.3}, I will
examine \citegen{Winford1993} analysis of property items from the
perspective of \isi{aspectual status} rather than \isi{categorial status}.  Here,
I will first turn to the broader issue of \isi{Stativity} as explored in the
literature on Caribbean Creoles.

\section{The question of the Stative/Non-stative
  distinction}\label{sec:3.2}

As mentioned in \chapref{ch:2}, data such as \xxref{ex:3:15}{ex:3:16} below which contain
the property items \textit{sik} `sick', \textit{weeri} `weary' and
\textit{redi}  `ready' have been called upon to question the validity
of the \isi{Stative}\slash Non-stative distinction and the notion of a unique
contribution of the verb to \isi{Aspect} in CECs:

\ea%15
\label{ex:3:15}

\ili{Guyanese Creole} (GC\il{Guyanese Creole}) \isi{Stative} usage\\\citep[31]{Jaganauth1987}
\ea
\gll  Mi \textbf{sik}.\\
\textsc{1sg} sick\\
\glt `I am sick.'


\ex
\gll Mi \textbf{weeri}.\\
\textsc{1sg} weary\\
\glt `I am weary.'

\ex
\gll Shi \textbf{redi}.\\
\textsc{3sg} ready\\
\glt `She is ready.'' \z \z

\ea%16
\label{ex:3:16}
GC\il{Guyanese Creole} \isi{Stative} verbs in Non-stative use \\\citep[31]{Jaganauth1987}
\ea
\gll Da tablit \textbf{sik} mi stomik.\\
	that tablet sick \textsc{1sg} stomach        \\
\glt `That pill has made me ill.'

\ex
\gll Dis baskit \textbf{weeri} mi.\\
     this basket weary \textsc{1sg}       \\
\glt `This basket has made me tired.'

\ex
\gll I \textbf{redi} shi.\\
\textsc{3sg} ready \textsc{3sg}\\
\glt `He has gotten her ready.' \z \z

In these examples, the lexical items \textit{sik} `sick',
\textit{weeri} `weary' and \textit{redi} `ready' appear in transitive
constructions with Non-stative meanings \REF{ex:3:16} and also in
intransitive constructions with \isi{Stative} meanings \REF{ex:3:15}.

Due to the existence of items such as these which appear in both
\isi{Stative} and Non-stative use, and also inconsistencies in Bickerton’s
observation of a clear difference in \isi{Tense} interpretation of \isi{Stative}
as opposed to Non-stative verbs authors are divided on the validity
of this distinction.  On the one hand, there are authors like
\citet{Winford1993} who adopt a position that ``certain predicators
involve change or process while others do not” (p.~34). In addressing
what he calls ``apparent inconsistencies” in the behaviour of lexical
items he explains that these ``can be accounted for without abandoning
the basic distinction between stative and non-stative verbal lexemes”
(p.~29).  Winford, explains variability in the behaviour of certain
lexical items in the context of

\begin{quote}
an ongoing process of decreolization, involving the apparent loss of
transitivity in some cases and categorial shift from more verbal to a
more truly adjectival status.” (p.~196)
\end{quote}

He posits that ``[t]he most convenient solution would be for each item
with a transitive function to be listed separately in the lexicon.”
\citep[196]{Winford1993}.  Without examining this position at this stage (see Chapters~\ref{ch:5} and~\ref{ch:6}
for discussion), I am in agreement with Winford that the
basic intuition associated with the \isi{Stative}\slash Non-stative distinction
need not be discarded.  Rather, the different Tense-\isi{Aspect}
interpretations of \isi{Stative} and Non-stative verbs point to a need to
better understand this distinction and how it works.  This is the
viewpoint that is also implied by authors such as \citet{Andersen1990}
and \citet{Gooden2008}.

In contrast to Winford’s position however, there are those authors who
point to a conceptual flaw in \citegen{Bickerton1975} claim of the
\isi{Stative}\slash Non-stative distinction as ``crucial'' in the Tense-\isi{Aspect}
system of creole languages.  The issue essentially is, if it is the
case that there are verbs that are \isi{Stative} and those that are
Non-stative, how does one account for lexical items that may be one or
the other?  Faced with the prospect of having listings in the lexicon
of identical items differentiated only on the basis of \isi{Stativity}, we
may recall that authors such as \citet{Sidnell2002} point to the
actual use of a verb as determining the aspectual meaning it denotes
rather than a lexical division of verbs.  According to
\citet[167]{Sidnell2002} it is ``necessary to categorize many verbs
according to their uses rather than according to some abstract lexical
specification\footnote{It is not clear what Sidnell means when he
  makes reference to ``\isi{lexical specification}''. This could refer to the
  semantic categories of \citet{Dixon1977} which \citet{Winford1993}
  uses in his categorisation. If this is the case, then I do agree
  with him that this is not sufficient. However, in reference to the
  lexical template of an item, I will explore in \chapref{ch:4} the
  notion of Event Structure and argue consistent with authors such as
  \citet{Pustejovsky1991} and \citet{Levin1993} that there is a part
  of the \isi{lexical specification} of an item that ``predicts the different
  uses that may be associated with a lexical item.}” (p.~167).  By
this he suggests a separation between the use of an item in context
and the \isi{lexical specification} of such a form; ultimately discarding
the latter.  This is the position of other authors in CECs such as
\citet{Jaganauth1987} who argues against a separation of verbs based
on the \isi{Stative}\slash Non-stative distinction due to the varying uses that
they may display (see \sectref{sec:Jaganauth} for full
discussion).

The problem with such approaches is that there is not a clear position
on how \isi{Aspect} as a cohesive system is to be treated in CECs.  For
example, items displaying the aspectual behaviour of those in
\REF{ex:3:15} and \REF{ex:3:16} above form a class, but we do not see
the same behaviour in all relevant forms in the language.  Although
these items fall within the larger class of property items, we will
see below in \sectref{sec:3.3} that not all items within this group
display the same flexibility in behaviour: not all (\isi{Stative}) property
items allow for a contrasting Non-stative version or what I treat as
the introduction of a causative or agentive element (\chapref{ch:5}). 
From this we see that there is a need to further understand
this group of predicates and how they fit into a general system of
\isi{Aspect}.  In other words, is there a mechanism that can account for the
behaviour of such items while also accounting for the behaviour of
clearly \isi{Stative} and Non-stative items?

To date, not many authors have attempted to address the case of
property items from the perspective of a holistic aspectual
categorization.  To my knowledge, \citet{Winford1993} is unique in
this respect as he addresses the entire group of property items rather
than individual items of interest.  In the following section, I will
spend some time reviewing the analysis of \citet{Winford1993}.  An
evaluation of his classification will point to the desirability of a
model which categorises items based on their aspectual behaviour
(syntactic and semantic criteria) rather than the semantic notions of
\citet{Dixon1977} which form the basis of Winford’s account.
 
\section{Winford’s semantic categorisation of CEC property
  items and an evaluation}\label{sec:3.3}
  
\subsection{Winford’s semantic categorisation of CEC property  items}\label{sec:3.3.1}
  
\tabref{extab:3:17} shows Winford’s categorisation of property items
in CECs based on semantic concepts ranging from more temporary to more
permanent (cf. \citealt{Dixon1977}).  As noted previously, Winford’s
analysis has been the only attempt at a generalisation over the
behaviour of property items in CECs on the grounds of aspectual
status.

\begin{sidewaystable}
  \caption{Property Concepts in CECs (adapted from \citealt[184]{Winford1993}).\label{extab:3:17}}
 \small
    \begin{tabular}{p{2.7cm}p{2cm}p{2cm}p{1.7cm}p{1.7cm}p{2.7cm}p{2cm}}
      \lsptoprule
      \multicolumn{7}{Z{17.5cm}}{more transitory \hfill $\xlrightarrow{6cm}$ \hfill  more permanent}  \\
      \tablevspace
      Physical property\newline (A) &   
                                      Dimension\newline (B) &
                                                              Colour\newline (C) &
                                                                                  Age\newline (D) & 
                                                                                                    Value\newline (E) &
                                                                                                                        Human\newline propensity (F) & 
                                                                                                                                                               Speed\newline (G)\\
      \midrule
      \textit{ded} `dead'

      \textit{drai} `dry'

      \textit{ful} `full'

      \textit{haad} `hard'

      \textit{hat} `hot'

      \textit{kool} `cold'

      \textit{raip} `ripe'

      \textit{saaf} `soft'

      \textit{sik} `sick'

      \textit{sowa} `sour’

      \textit{swiit} `sweet'

      \textit{wet} `wet'

                                    & 
                                      \textit{big} `big'

                                      \textit{braad} `broad'

                                      \textit{fain} `fine'

                                      \textit{fat} `fat'

                                      \textit{lang} `long'

                                      \textit{maaga} `slim'

                                      \textit{shaat} `short'

                                      \textit{smaal} `small'

                                      \textit{taal} `tall'

                                      \textit{waid} `wide'

                                                            &
 
                                                              \textit{blak} `black’

                                                              \textit{daak} `dark’

                                                              \textit{griin} `green’

                                                              \textit{red} `red'

                                                              \textit{wait} `white'

                                                              \textit{yela} `yellow' 
                                                                                & 

                                                                                  \textit{njuu} `new'

                                                                                  \textit{ool} `old'

                                                                                  \textit{yong} `young'

                                                                                                  &
                                                                                                    \textit{bad} `bad'

                                                                                                    \textit{gud} `good'

                                                                                                    \textit{nais} `nice'
                                                                                                                      &
                                                                                                                        \textit{chupid} `stupid'

                                                                                                                        \textit{hapi} `happy'

                                                                                                                        \textit{jelas} `jealous'

                                                                                                                        \textit{leezi} `lazy'

                                                                                                                        \textit{mad} `mad'

                                                                                                                        \textit{ruud} `rude'

                                                                                                                        \textit{wikid} `wicked'

                                                                                                                        \textit{wotlis} `worthless' 
                                                                                                                                                             &
                                                                                                                                                               \textit{faas} `fast'

                                                                                                                                                               \textit{kwik} `quick'

                                                                                                                                                               \textit{sloo} `slow'\\
      \multicolumn{3}{l}{most vulnerable to dual aspectual interpretations} && 
                                                                                \multicolumn{3}{r}{least vulnerable to dual aspectual interpretations}\\
      \tablevspace
      inherently denote change of state & \multicolumn{6}{|c}{do not inherently denote change of state}\\
      \lspbottomrule
    \end{tabular}
\end{sidewaystable}


As shown in \tabref{extab:3:17}, CEC property items may be classified
based on semantic concepts ranging from those that are more temporary
(i.e: Physical Property, Dimension and Colour) to those that are more
permanent (Age, Value etc).  \citet{Winford1993} observes that the
items most vulnerable to what I call dual aspectual interpretations
are those denoting ``\isi{transitory states}'' such as Physical Property,
Dimension and Colour as opposed to those which represent more
``permanent qualities'' such as Age, Value, etc. (p.~184).  Based on the
ability of items to appear with \isi{Progressive} \isi{Aspect} marking and their
ability to appear in transitive usage, he observes a general split in
this semantic classification between items expressing Physical
Property and all others (p.~187).\largerpage[-2]

In his analysis, items which express Physical Property ``behave rather
like Change of state\is{State!Change of} (process) verbs whose semantic features are
compatible with \isi{Progressive} aspect”.\footnote{While I do not accept the
  compatibility of an item with \isi{Progressive} aspect as a test for the
  \isi{Stativity} of the form itself, the meaning that results where this
  interaction takes place provides insights into the inherent
  \isi{aspectual status} of the items (see discussion in \chapref{ch:5},
  \sectref{sec:5.4}). Thus for example with regular Non-stative verbs,
  we may note that the interpretation with the \isi{Progressive} is
  generally that of an ongoing process with no Change of state\is{State!Change of} implied
  while in the case of \isi{dual aspectual forms} a Change of state\is{State!Change of}
  interpretation comes into focus. (cf. examples in \REF{ex:3:18}
  below).} Hence, he labels them ``essentially Non-stative” in
comparison to items which express the semantic concepts of Dimension,
Colour, Human Propensity which, according to him, ``behave rather like
\isi{Stative} verbs” (p. 187--188).  Thus, Winford’s model predicts the
\isi{aspectual status} of CEC property items based on semantic concepts.
His observation of a split in the \isi{Stativity} of these items coincides
with my own observation of some items being essentially Non-stative
while others are \isi{Stative}.  However, the predictions which his model
makes are not borne out, as I will show below.  Moreover, I differ in
my characterisation of inherently \isi{Stative} items as adjectives as
opposed to Winford’s characterisation of these items as (\isi{Stative})
verbs (see \chapref{ch:6} for discussion).

\subsection{An evaluation}\label{sec:3.3.2}

In this section I will evaluate Winford’s model based on some JC\il{Jamaican Creole}
data. Of interest here is whether or not items falling into a
particular \isi{semantic category} display the behaviour predicted.  Recall
that Winford cites a separation between property items of the semantic
type Physical Property as Non-stative (i.e.: compatible with
\isi{Progressive} aspect) as opposed to items in other categories which are
\isi{Stative}.  Here, I will attempt to highlight the basics of this
separation based on some JC\il{Jamaican Creole} data.  It will become evident that while  there are items which fit the pattern that Winford suggests in terms
of \isi{Stativity}, there are also items which are inconsistent with the
expected behaviour of their semantic group.

Starting with the group of items in category A (Physical Property), an
item such as \textit{ded} `dead' in JC\il{Jamaican Creole} may be shown to be typical of this group in terms of its ability to express Non-stativity and in
particular a Change of state\is{State!Change of}.  The way in which this interpretation is
manifested varies as it may be contextually achieved (\ref{ex:3:17}a--b),
expressed through the use of a temporal adverbial (\ref{ex:3:17}c) or
evident in the use of \isi{Imperfective} aspect marking (\ref{ex:3:17}d).
Note that all JC\il{Jamaican Creole} data in this section (except where otherwise
indicated) are drawn from my personal intuitions).\largerpage[-1]

\ea%17
\label{ex:3:17}
\ili{Jamaican Creole}  (JC\il{Jamaican Creole})\\
  \ea
 	 \gll Di man \textbf{ded.}\\
  \textsc{art} man dead        \\
    \ea  `The man is dead.'\\
    \ex  `The man died.'\\
    \z 

  \ex
  \gll Di man \textbf{ded} iina di aksident.\\
\textsc{art} man dead in the accident    \\
  \glt `The man died in the accident.'

  \ex
  \gll Di man \textbf{ded} sed spiid.\\
  \textsc{art} man dead same speed      \\
  \glt `The man died immediately.'

  \ex
  \gll	  Di man      	   a \textbf{ded}.\\
  \textsc{art} man \textsc{asp} dead        \\
  \glt `The man is dying.'
  \z
\z

The JC\il{Jamaican Creole} item \textit{ded} `dead' is shown here as ambiguous between a \isi{Stative} and Non-stative interpretation in (\ref{ex:3:17}a) and clearly Non-stative in its interpretation in (\ref{ex:3:17}b--d).  Winford
labels items displaying the behaviour of \textit{ded} `dead' and
falling into the category of items expressing Physical Property as
Non-stative (Change of state\is{State!Change of}).

Two questions arise from these observations.  The first has to do with
explanatory adequacy, and is the question of the labeling of an item
which is able to express both \isi{Stativity} and Non-stativity as
``essentially non-stative”, as Winford does (p.~184).  This is an issue
that I will deal with in \chapref{ch:4} where I attempt to elucidate the
notion of Change as associated with the \isi{Stative}\slash Non-stative
distinction and the conceptual question of how an item may be
associated with both \isi{Stativity} and Non-stativity.  The second question
has to do with \isi{observational adequacy} given Winford’s attempt at a
generalisation over the behaviour of items based on their semantic
class.  Thus the question is whether or not all items falling into the
group of Physical Property can be shown to display similar behaviours
in terms of their ability to appear in Non-stative use.  This is the
question that I will deal with here.

Without paying attention to the specifics of how an item allows for
the expression of Non-stativity\footnote{In \chapref{ch:5} I outline
  that Non-stativity may be indicated through the presence of
  \isi{Imperfective} aspect marking, temporal adverbials and also in the
  case of the causative\slash inchoative alternation. However, there may be
  lexical items which show variation in their acceptance of all these
  contexts even where they allow for Non-stative expression. Thus,
  there may be an item which allows for \isi{transitive variation} but does
  not allow for \isi{Imperfective} aspect marking; in this way it allows for
  an interpretation that is Non-stative but resists \isi{Imperfective}
  aspect marking. In this work, the focus is not on the range of
  Non-stative expression that is allowed but rather the fact that an
  item allows for such expression and the meaning that is indicated in
  such an instance.}, it is apparent that there are items in JC\il{Jamaican Creole} which, based on \citet{Winford1993}, fall into the semantic class of Physical
Property but which are resistant to the expression of Non-stativity.
Items such as \textit{saaf} `soft', \textit{haad} `hard'\textit{,
  swiit} `sweet', \textit{sowa} `sour', etc., may be said to be
a-typical of property items expressing Physical Property in that they
do not appear to be compatible with Non-stative meaning.  Compare:

\ea%18
\label{ex:3:18}
\ili{Jamaican Creole} \\
\ea[]{
\gll Di mango \textbf{saaf}.\\
\textsc{art} mango soft      \\
\glt `The mango is soft.'}

\ex[*]{
\gll Di mango a \textbf{saaf}.\\
\textsc{art} mango \textsc{asp} soft    \\
\glt  `The mango is getting soft.'\\}

\ex[*]{
\gll Dem \textbf{saaf} di mango.\\
\textsc{3pl} soft \textsc{art} mango  \\
\glt `They are softening/making the mango soft.'} \z \z


\ea%19
\label{ex:3:19}
\ili{Jamaican Creole}\\
\ea[]{
\gll Di    siment \textbf{haad}.\\
\textsc{art} cement hard      \\
\glt `The cement is hard.'}

\ex[*]{
\gll  Di siment a \textbf{haad}\\
\textsc{art} cement \textsc{asp} hard    \\
\glt `The cement is hardening.'}

\ex[*]{
\gll Dem \textbf{haad} di siment.\\
\textsc{3pl} hard the cement    \\
\glt `They made the cement hard.'} \z \z

\ea%(20)
\ili{Jamaican Creole}\\
\ea[]{
\gll  Di lemanied \textbf{swiit}.\\
\textsc{art} lemonade sweet    \\
\glt `The lemonade is sweet.'}

\ex[*]{
\gll Di lemanied a \textbf{swiit}.\\
\textsc{art} lemonade \textsc{asp} sweet    \\
\glt `The lemonade is getting sweet.'}

\ex[*]{
\gll Dem \textbf{swiit} di lemanied.\\
     \textsc{3sg} sweet the lemonade  \\
\glt `They sweetened the lemonade/made the lemonade sweet.'} \z \z

\ea%21
\label{ex:3:21}
\ili{Jamaican Creole}\\
\ea[]{
\gll Di juus \textbf{sowa}.\\
\textsc{art} juice sour\\
\glt `The juice is sour.'}

\ex[*]{
\gll Di juus \textbf{a} \textbf{sowa}.\\
\textsc{art} juice \textsc{asp} sour      \\
\glt `The juice is getting sour.'}

\ex[*]{
 \gll Dem    pikni / laim \textbf{sowa} di juus.\\
\textsc{pl} child / lime sour the juice  \\
\glt `The children/limes made the juice sour.'} \z \z

The examples (\ref{ex:3:18}--\ref{ex:3:21}) show the items
\textit{saaf} `soft', \textit{swiit} `sweet', \textit{sowa} `sour’ as
resistant to Non-stative interpretations\footnote{Non-stative
  expression for such items are available through the use of the
  \isi{semi-copula} form \textit{get} or by means of the morphological
  operation which adds the suffix -\textit{op} thus creating a complex
  morphological verb (cf.: \textit{di mango saaf-op\slash di mango get
    saaf} `the mango got soft') however, these have no immediate
  relevance to this discussion.} in contrast to an item such as
\textit{ded}  `dead' which presumably falls in the same semantic
category.  This difference in aspectual behaviour suggests an
aspectual split in the category of items semantically expressing
Physical Property: while some are open to Non-stative expression,
others are not.

In the case of items displaying the behaviour of \textit{saaf} `soft',
\textit{swiit} `sweet', \textit{sowa} `sour',  etc. these may be said to
form a natural class with other items categorized as semantically
expressing Age, Value, Human Propensity and Speed which based on
Winford’s model are ``essentially \isi{Stative}” in that they are
incompatible with \isi{Imperfective} aspect in JC\il{Jamaican Creole}.  Items in these semantic classes seem for the most part to denote \isi{Stativity} but as with the case of Physical Property items, there are some exceptions which
suggest a need to look more closely at the categorisation presented by
Winford. The examples below show lexical items from the categories
Dimension, Colour, Age Human Propensity and Speed which display
behaviour typical of these groups since, as indicated, these are
expected not to be compatible with Non-stative meaning:

\ea%22
\label{ex:3:22}
(Dimension) \ili{Jamaican Creole} \\


\ea[]{
\gll Di riva \textbf{waid/lang/braad.}\footnotemark{}\\
\textsc{art} river wide/long/broad    \\
\footnotetext{As will be discussed in \chapref{ch:5}, categorisation of
  these items may differ across speech communities; note for example
  that the similar form \textit{bradi} in \ili{Sranan} may appear in both
  \isi{Stative} and Non-stative use. Also, there are indications of
  variability in its behaviour in JC\il{Jamaican Creole} as well.}  
\glt `The river is wide\slash long\slash broad.'}

\ex[*]{
\gll Di riva        a \textbf{waid/lang/braad}.\\
\textsc{art} river \textsc{asp} wide/long/broad  \\
\glt `The river is widening\slash lengthening\slash broadening.'}

\ex[*]{
\gll Dem a \textbf{waid/lang/braad} di riva.\\
\textsc{3pl} \textsc{asp} wide/long/broad \textsc{art} river  \\
\glt `They are widening\slash lengthening/broadening the river.'}

\ex[*]{
\gll Di    uman       a \textbf{fat}.\\
\textsc{art} woman \textsc{asp} fat      \\
\glt `The woman is getting fat.'} \z \z

\ea%23
\label{ex:3:23}
(Colour) \ili{Jamaican Creole}\\
\ea[]{
\gll Di graas \textbf{griin}.\\
\textsc{art} grass green        \\
\glt `The grass is green.'}

\ex[*]{
\gll Di graas   a \textbf{griin}.\\
\textsc{art} grass \textsc{asp} green      \\
\glt `The grass is getting green.'}

\ex[*]{
\gll Mi faada \textbf{griin} di graas.\\
\textsc{1sg} \textsc{poss} father greean \textsc{art} grass    \\
\glt `My father made the grass green.'} \z \z

\ea%24
\label{ex:3:24}
(Value) \ili{Jamaican Creole} \\
\ea[]{
\gll di plies \textbf{nais}.\\
\textsc{art} place nice          \\
\glt `The place is nice.'}

\ex[*]{
\gll Di plies \textbf{a} \textbf{nais}.\\
\textsc{art} place \textsc{asp} nice  \\
\glt `The place is becoming nice.'}

\ex[*]{
\gll Dem \textbf{nais} di plies.\\
\textsc{3pl} nice \textsc{art} place  \\
\glt `They made the place (look) nice.'} \z \z

\ea%(25)
(Age)    \ili{Jamaican Creole}\\

\ea[]{
\gll Mi kluoz dem \textbf{uol}.\\
        \textsc{1sg} clothes \textsc{pl} old          \\
\glt `My clothes are old.'}

\ex[*]{
\gll  Mi kluoz dem \textbf{a} \textbf{uol}.\\
\textsc{1sg} clothes \textsc{pl} \textsc{asp} old        \\
\glt `My clothes are getting old.'}

\ex[*]{
\gll Mi sista \textbf{uol} mi kluoz.\\
\textsc{1sg} sister old \textsc{1sg} clothes  \\
\glt `My sister made my clothes old.'}

\z \z


\ea
% (26)
(Human Propensity)    \ili{Jamaican Creole}\\
\ea[]{
\gll Da man de \textbf{chupid}.\\
that man     \textsc{loc}   stupid        \\
\glt `That man is stupid.'}

\ex[*]{
\gll Da man  de \textbf{a} \textbf{chupid}.\\
that man \textsc{loc} \textsc{asp} stupid  \\
\glt `That man is getting\slash behaving stupid.'}

\ex[*]{
\gll Di uman \textbf{chupid} di man.\\
the woman stupid \textsc{art} man  \\
\glt `The woman made the man stupid.'} \z \z

\ea%27
\label{ex:3:27}
(Speed) \ili{Jamaican Creole}\\

\ea[]{
\gll Di kontri bos dem \textbf{sluo}.\\
the country bus \textsc{pl} slow  \\
\glt `The buses from the rural areas are slow.'}

\ex[*]{
\gll Di kontri bos dem \textbf{a} \textbf{sluo}.\\
          the country bus \textsc{pl} \textsc{asp} slow  \\
\glt `The buses from the rural areas are getting slow.'}

\ex[*]{
\gll Di bad ruod dem \textbf{sluo} di kontri bos dem.\\
		the bad road \textsc{pl} slow the country bus \textsc{pl}  \\
\glt `The bad roads make the buses from the rural areas slow.'} \z

\z

The examples \xxref{ex:3:22}{ex:3:27} above show JC\il{Jamaican Creole} items \textit{waid\slash lang\slash braad}
`wide\slash long\slash broad', \textit{griin} `green', \textit{uol} `old',
\textit{nais} `nice', \textit{chupid} `stupid' and \textit{sluo} `slow'
which fall into the categories Dimension, Colour, Age, Value, Human
propensity and Speed respectively in \isi{Stative} use.  As shown as well,
these items are defiant to Non-stative expression as shown in the (b)
and (c) examples where attempts at \isi{Imperfective} aspect and transitive
variation are made.  Although they may express this by means of the
\isi{semi-copula} form \textit{get} or by means of the morphological
operation which adds the suffix -\textit{op} or \textit{dung}
  (up\slash down) thus creating a \isi{complex morphological verb}, this is not
immediately relevant to the present discussion and will not be
addressed.  Essentially, what these examples indicate is that such
items at least in JC\il{Jamaican Creole} are not conceived as inherently involving a Change of state\is{State!Change of}.

Based on Winford’s classification, the behaviour of items from these
categories is not particularly noteworthy as this is consistent with
his expectations.  However, as in the case of items falling into the
category expressing Physical Property, there are some notable
exceptions that question the \isi{observational adequacy} of Winford’s
model.  Items of interest in this regard include \textit{blak}
`black', \textit{red} `red' from the group expressing Colour,
\textit{bad/ruud} `bad\slash rude' from the category expressing Value\slash Human
Propensity, and \textit{jelas} `jealous', and \textit{mad} `mad', from
the group expressing Human Propensity.  As shown below, these items
allow for Non-stative interpretations to varying degrees in contrast
to what may be said to be the typical behaviour of items in their semantic group. 
Compare:

\ea%28
\judgewidth{??}
\label{ex:3:28}
\ili{Jamaican Creole} (Colour) \\
\ea[]{
\gll di skert \textbf{red/blak}.\\
\textsc{art} skirt red/black      \\
\glt `The skirt is red\slash black.'}

\ex[??]{
\gll di skert    a \textbf{red/blak}.\\
\textsc{art} skirt   \textsc{asp} red/black      \\
\glt `The skirt is getting red\slash black.'
}

\ex[??]{
\gll dem \textbf{red/blak} di skert.\\
        \textsc{3pl} red/black \textsc{art} skirt  \\
\glt `They made the skirt red\slash black.'
\glt also `They reddened\slash blackened the skirt.'} \z \z

As indicated here, items such as \textit{red} `red', and \textit{blak}
 `black' from the category expressing Colour may be marginally
acceptable in Non-stative use in JC\il{Jamaican Creole}. However, there are particular instances where these items are clearly acceptable in Non-stative use.
Compare \REF{ex:3:28d} which was heard uttered in a context where
the sun was attributed with the change in colour seen in a mango on a
tree as opposed to a natural state of ripeness:

\ea%29
\label{ex:3:28d}
  \gll   Di  son \textbf{red} di mango.\\
\textsc{art} sun         red \textsc{art} mango  \\
\glt `The sun reddened the mango\slash made the mango red.' \z

In Non-stative use, the interpretation of items such as ‘red’ and
 `black' is that of a Change from one \isi{State} to another.  I will discuss
the semantic implications of this in \chapref{ch:5}.  It will become
apparent that the Non-stative meaning indicated by these items is
distinct from that which is expressed through items of the type
\textit{ded} `dead', suggesting that there is a need to pay closer
attention to the semantic interpretation that arises as opposed to
simply the fact that an item appears in Non-stative use.

The examples in \REF{ex:3:29} show another instance where Non-stative
interpretation is allowed (unexpectedly based on Winford’s model), and
where the interpretation is distinct from that of a Change of state\is{State!Change of},
which represents Winford’s Non-stativity:

\ea%29
\label{ex:3:29}
(Value\slash Human Propensity)  \ili{Jamaican Creole}\\
\ea
\gll Da pikni      de      \textbf{bad/ruud}. \\
	that child   {\DEM}            bad/rude      \\
\glt `That child is (a) bad\slash rude (child).'

\ex
\gll Dat pikni     de            a \textbf{bad/ruud}     (lang taim).\\   
  	that child     {\DEM} \textsc{asp}     bad/rude {\db}long time  \\
    \glt `That child has been misbehaving for a long time.' \z \z

Here we see items such as \textit{bad} `bad' and \textit{ruud} `rude'
appearing with Non-stative interpretations but not expressing a Change
from one \isi{State} to another.  In fact, what may be said to be expressed
in the Non-stative use of these items is \isi{Process} which does not result
in a Change of state\is{State!Change of} (i.e.: action associated with being bad or rude).
This is similar to the meaning expressed by \textit{jelas} `jealous'
in \REF{ex:3:30}:\pagebreak

 
\ea%30
\label{ex:3:30}
\ili{Jamaican Creole}\\
\ea
\gll Dem jealous.\\
\textsc{3pl} jealous              \\
\glt `They are jealous.'

\ex
\gll    Dem          a \textbf{jelas}         mi   fi            di   kyar   we            mi jraiv.\\
\textsc{3pl} \textsc{asp}     jealous \textsc{1sg} for \textsc{art}    car   that \textsc{1sg} drive  \\

\glt `They are (being) jealous\slash envious of me because of the car that I drive.' \z \z

In the example in (b) the lexical item \textit{jelas} ‘jealous’
appears in Non-stative use to signal a meaning which points to actions
associated with a particular state (jealousy\slash envy).  Such usage is not
accounted for in Winford’s categorisation of property items which
distinguishes Change of state\is{State!Change of} verbs and \isi{Stative} verbs.

The case of \textit{mad} `mad’ which also falls into the category of
Human Propensity, is similarly able to appear in Non-stative use:

\ea%31
\label{ex:3:31}
(Human Propensity) \ili{Jamaican Creole}\ea
  \gll Im \textbf{mad}.\\
\textsc{3sg} mad          \\
\glt `He is crazy.'

\ex
 \gll  im           a \textbf{mad}.\\
\textsc{3sg} \textsc{asp}      mad        \\
\glt `He is going crazy.'

\ex
\gll A uman \textbf{mad}        im.\\
	is woman        mad \textsc{3sg}    \\
\glt `(It’s) A woman (that) drove him crazy.' \z \z

As shown in \REF{ex:3:31}, \textit{mad} `mad' appears in the full range of
Non-stative uses; allowing for both \isi{Imperfective} aspect and transitive
use.  This, again, is a-typical of items in this \isi{semantic category},
which based on Winford’s model, are expected to be \isi{Stative} and ``behave
like stative verbs” (p.~188).  What the examples in \xxref{ex:3:29}{ex:3:31} point to is a level of variation in the expression of \isi{Stativity} among property
items that is not just across semantic categories but even within these.

The aspectual behaviour evident in this group of items, based on this
brief examination of JC\il{Jamaican Creole} data, indicates that there is more variation among property items than Winford noted.  The extent of this variation
does not allow for a treatment of particular items as exceptions to
the rule but rather points to the need for a different model.  In
particular, it may not be possible to generalise over the behaviour of
lexical items in these groups -- at least not based on their membership
in semantic classes Winford distinguishes.  I propose to observe
specific behaviours and focus on accounting for these from an Event
Structure perspective.


\section{Summary of observations}\label{sec:3.4}

The problem presented by property items and \isi{dual aspectual forms} in
particular may be summed up as follows: the diversity of this group of
items.  This is apparent in both the discussion of the aspectual and
\isi{categorial status} of these items.  As it relates to \isi{categorial status},
items appear in both verbal and adjectival uses leading authors to
seek an account that selects one category while explaining away the
other (cf. \citealt{Sebba1986,Seuren1986}).  Others may, as
\citet{Kouwenberg1996} does, posit a status for this group of items
that includes both verbs and adjectives.

The former approaches have the immediate drawback of analysing this
group of items as monolithic (verbs or adjectives), thereby not
recognising the diversity that is apparent in their behaviour.  From
this perspective, these may be judged to be more subjective in their
analyses where \citet{Kouwenberg1996} may be seen as more objective in
her treatment since she recognises this diversity and attempts to
account for it.  Nevertheless, none of these approaches achieve an
understanding of what it is that accounts for the variation in
behaviour that we see within this group of items.

In terms of the \isi{aspectual status} of these items, it is apparent first
of all that within the context of the \isi{Stative}\slash Non-stative distinction,
there is a large number of items which appear in Non-stative use along
with their \isi{Stative} uses.  While \citet{Winford1993} claims a split in
the \isi{aspectual status} of these items based on semantic groups (i.e.,
items expressing Physical Property as Non-stative as opposed to all
others as \isi{Stative}), observation of JC\il{Jamaican Creole} data shows that items which appear in Non-stative use are not restricted to the semantic group of
Physical Property as indicated by Winford.  Rather, items displaying
this flexibility in usage range across his semantic categories and
display different subtypes of Non-stative meaning, including Change of
state (e.g., \textit{ded} `dead', colour items), and \isi{Process} (e.g.,
\textit{ruud} `rude', \textit{bad} `bad', \textit{jelas} `jealous').


The different aspectual interpretations associated with property items
in Non-stative use indicate that we are dealing with different types
of items.  Based on the preliminary evaluation above, it seems that
there are at least three different classes of items in this group:
First, items of the type \textit{ded}  `dead' and also those expressing
Colour like \textit{red} `red' and {\textit{blak}} `black' which appear in \isi{Stative} and Non-stative use and which, in the latter use, indicate a 
Change of state\is{State!Change of}.  Secondly those of the type \textit{chupid} `stupid',
\textit{saaf} `soft', \textit{haad} `hard', \textit{swiit} `sweet'
etc., which do not appear in Non-stative use.  And thirdly those of the
type \textit{jelas} `jealous' and \textit{bad} `bad' which, like the
first group of items, appear in both \isi{Stative} and Non-stative use, but
do not indicate a Change from one state to another but rather an
ongoing \isi{Process}.  \chapref{ch:4} will set the background for the
classification of these items based on an Event Structure that I will
elaborate in subsequent chapters.

In the treatment that I espouse the question of the \isi{categorial status}
is addressed as secondary in relation to \isi{aspectual status}. However, as
we will see in Chapters~\ref{ch:5} and~\ref{ch:6} in particular, there is a logical
relation between the two issues, such that an understanding of
aspectual behaviour provides insights into \isi{categorial status}. In the
chapters which follow, I will seek to account for the diversity that
is expressed in the behaviour of property items.  First by exploring
what is indicated by the \isi{Stative}\slash Non-stative distinction and what
would account for a single item which is able to express both these
meanings.  It will become apparent that one may not be able to
generalise over the behaviour of specific property items across
Creoles.  Nevertheless, there is a level at which the categorisation
of an item and its actual behaviour can be understood in an
appropriate model based on event types.  From such a perspective it
may be possible to offer an account of CEC property items from a more
universal perspective while allowing for a language-specific
categorisation of these items.
