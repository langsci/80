\chapter[Syntactic behaviour, event types and semantic interpretations]{Towards a model for the classification of property items:
Syntactic behaviour, event types and semantic interpretations}
 \label{sec:5}\label{ch:5}

\section{Background}\label{sec:5.0}

In \chapref{ch:3}, I outlined the problem presented by \isi{dual aspectual forms} in CECs. Assuming, as I do, that the verb makes a unique aspectual contribution to \isi{Aspect}, we are faced here with the need for an explanation of the aspectual behaviour of these items in CECs. In essence, how do we account for the fact that a single lexical item can express different aspects if it is the case that each item is associated with a unique aspectual value? In \chapref{ch:4}, following \citegen{Comrie1976} account of the \isi{Stative}\slash Non-stative distinction, I identified the notion of Change as the distinguishing feature separating verbs that express Change from those that do not. Further to this, consistent with discussion of works such as that of \citet{Levin1993}, and the earlier works of \citet{McCawley1968,Carter1976,Dowty1979,Pustejovsky1988,Grimshaw1990}, I point to Change as a lexico-semantic concept indicated by primitive elements of meaning such as \BECOME, \CAUSE and \DO. 

The notion of Change as a semantic concept associated with some verbs and not with others is appealing in that it may, in the case of CECs, account for the default \isi{Tense} interpretation of \isi{Stative} versus Non-\isi{Stative} verbs as recognised in the work of \citet{Bickerton1975,Winford1993,Gooden2008} among others (see the discussion in \chapref{ch:3}). Nonetheless, as we have seen, this does not account for the case of CEC property items that may be conceived as expressing Change in some instances but not in others. The question is, if Change is a semantic feature of the verb, are items such as these both [+Change] and [\textminus Change] at the same time or are they one or the other? These are the questions that authors on CECs such as \citet{Bickerton1975,Jaganauth1987,Winford1993,Sidnell2002,} etc. have grappled with.\largerpage[-1] 

In this chapter, I will focus on a categorisation of property items based on the aspectual behaviour that they display. This will result in an alternative classification to that presented by \citet{Winford1993} which, as indicated in \chapref{ch:3}, is to my knowledge the most complete attempt at treating this group of items from the perspective of aspect. \citet{Winford1993} approaches CEC property items along the lines of \citegen{Dixon1977} semantic categories, positing a division between items expressing Physical Property as Non-stative and all others as \isi{Stative}. According to him, 

\begin{quote}Items expressing Physical Property behave rather like Change of state\is{State!Change of} (process) verbs whose semantic features are compatible with \isi{Progressive} aspect. Such verbs are essentially Non-stative. By contrast, it seems that items expressing the concepts associated with semantic types like Dimension, Colour, Human Propensity etc., behave rather like \isi{Stative} verbs. (p. 187--188)\end{quote}

The basic intuition, that there are two different types of property items in terms of \isi{Stativity}, is correct but it does not fully account for certain facts. In the first instance, Winford does not actually account for the fact that there are items across his semantic categories that display \isi{dual aspectual behaviour}, i.e., appear in both \isi{Stative} and Non-stative use. Nor does he address the semantics which allow for an item to be labeled \isi{Stative}\slash Non-stative especially given the fact that some appear in \isi{dual aspectual use}. And, in his attempt to treat property items across CECs, he does not account for the fact that there may be cross-linguistic variation in the aspectual behaviour of specific items. Recall as well that the members of his semantic classes do not all behave as predicted. In these regards, such a treatment may be said to be lacking in terms of both observational and explanatory adequacy.

Here, I will present a model for the analysis of property items in CECs based on syntactic and semantic criteria and the behaviour exhibited by JC\il{Jamaican Creole} items as discussed in \chapref{ch:3}. This model takes into consideration cross-linguistic variation that may be observed in the aspectual behaviour of lexical items. Such variation is attributable to the culturally based lexico-semantic conceptualisation of specific items, allowing for some to be treated as either \isi{State}, Change of state\is{State!Change of} or \isi{Process}, depending on the pertinent language variety. Owing to the lexico-semantic differences that may exist, I will not attempt a general classification for specific lexical items across CECs as Winford does but will focus on \ili{Jamaican Creole} (JC\il{Jamaican Creole}) for a classification -- which may serve as a model for the classification of similar forms in other CECs. 



I will show that there are two distinct classes of property items: The first is of the type \textit{sik,} which appears as the \isi{Stative} form `sick', the inchoative `get sick' and the transitive Non-stative `make sick'. Items of this type will be classified as Class 1 items and identified based on their ability to express a Change of state\is{State!Change of} from one (logical) opposition\footnote{This notion is elaborated in \sectref{sec:5.1.3}} to the next in their Non-stative use. Consistent with \citet{Pustejovsky1988,Pustejovsky1991}, I will argue that the members of this class have the Event Structure of \isi{Transition}. The second group, Class 2, is the class of items that are inherently States; but several subclasses may be distinguished within this class. On one hand, there are items of the type \textit{nyuu} `new', \textit{wotlis} `worthless', \textit{chupid} `stupid' which are \isi{Stative} items and do not vary in \isi{Stativity} (Class 2a). On the other hand there are those items which are also essentially \isi{Stative} but which I argue may be derived to express Non-stativity. In this latter use, they are characterised by the Event Structure of either \isi{Transition} (Change of state\is{State!Change of}) or \isi{Process}. These include items such as JC\il{Jamaican Creole} \textit{blak} `black'\slash `become black'\slash `cause to become black'\slash, \textit{red} `red'\slash`become red'\slash`cause to become red'\slash (Class 2b\footnote{As we will see in my discussion in \sectref{sec:5.2}, items such as these in their use as colour references in JC\il{Jamaican Creole} may be somewhat resistant to this kind of Non-stative use. However where they appear Non-statively, they express a Change of state\is{State!Change of}. They also appear in idiomatic uses with the meaning `sooty’ in the case of \textit{blak} `black’ and `burnt’ in the case of \textit{red.}}), vs. \textit{jelas} `jealous'\slash`behave jealously', \textit{bad} `bad'\slash`behave badly\slash misbehave' (Class 2c). 


The chapter is organised as follows: In \sectref{sec:5.1}, I will outline the criteria for the categorisation of property items and a conceptual descriptive model based on Event Structure as introduced in \chapref{ch:1} and further elaborated in \chapref{ch:4}. In \sectref{sec:5.2}, I will apply this model to a specific categorisation of property items in JC\il{Jamaican Creole}. I sum up my observations in \sectref{sec:5.3}. 

\section{Criteria for the categorisation of property items}\label{sec:5.1}

\subsection{Non-stative use: The progressive criterion}\label{sec:5.1.1}

In the study of CECs as well as elsewhere, \citegen{Vendler1967} \isi{Progressive} criterion has been used to evaluate \isi{Stativity} (see \citealt{Jaganauth1987,Bickerton1975,Gooden2008}). The assumption generally is that the presence of \isi{Imperfective} or \isi{Progressive} aspect marking points to Non-stativity while its incompatibility with a predicate indicates that such a predicate is \isi{Stative}. However, the well attested occurrence of \isi{Progressive} aspect marking with verbs accepted as expressing \isi{Stativity} (see \citealt{Verkuyl1993, Lyons1977,Smith1983,Smith1991}, etc.) makes it difficult to accept the \isi{progressive criterion} in and of itself as a reliable test for the \isi{inherent aspect} of a verb. In \chapref{ch:2}, we saw that similar problems arise in the use of this criterion in the discussion of CECs (see \sectref{sec:2.1.3}). 

In this case however, the \isi{progressive criterion} allows for the evaluation of the type of Non-stative meaning that arises where \isi{Progressive} aspect is allowed to interact with a predicate. Thus we note for example a difference in interpretation where \isi{Progressive} aspect interacts with the verb \textit{have} as opposed to \textit{run} and \textit{close} as shown below (Examples are recalled from \chapref{ch:1}, \sectref{sec:1.2.1}):


\ea%1
 \label{ex:5:1} 
 (adapted from \citealt[707]{Lyons1977})

\ea She \textbf{has} a headache. (\isi{Stative})
\ex She \textbf{is having} a headache. (Non-stative)
\ex She \textbf{is having} one of her headaches. (Non-stative)
\z
 \z

\ea%2
 \label{ex:5:2}
 John \textbf{is running.} (Non-stative)
 \z

\ea%3
 \label{ex:5:3}
\ea The door is \textbf{closed.} (\isi{Stative})
\ex The door \textbf{is closing.} (Non-stative)
\z
 \z

In \REF{ex:5:1}, the presence of \isi{Progressive} aspect with the \isi{Stative} verb \textit{have} enforces a viewpoint that is compatible with Non-stativity, namely a Processual viewpoint. Examples such as these have led authors such as \citet{Lyons1977,Smith1983} to pay attention to the meaning that arises in such instances rather than simply citing a restriction between \isi{Progressive} aspect and a verb indicating \isi{Stativity}. In particular, \citet{Lyons1977} points to a restriction between \isi{Stative} meaning and \isi{Progressive} meaning (p.~707). 

On a first examination of the examples in \xxref{ex:5:1}{ex:5:3} it would appear that the \isi{Progressive} serves the same purpose, namely it induces an ongoing \isi{Process} interpretation. However, upon closer examination, there are differences in the interpretation of the verbs in \REF{ex:5:1}, \REF{ex:5:2} and \REF{ex:5:3} as they interact with the \isi{Progressive}. In the case of (\ref{ex:5:1}b--c) the use of the \isi{Progressive} seems to extend the \isi{State} \textit{have a headache} by establishing a viewpoint of this situation as ongoing. Thus, both \textit{having a headache} and \textit{running} as in \REF{ex:5:2} may be taken to have occurred and in the context of the \isi{Progressive} ongoing. However, in the case of the door \textit{closing} (\ref{ex:5:3}b), this has not been achieved. In other words, \textit{She is having a headache} entails that \textit{She has a headache}. Also, \textit{John is running} entails: \textit{John ran}, however, \textit{The door is closing} does not entail \textit{the door has closed.} 

Based on the interpretations that arise in the examples \xxref{ex:5:1}{ex:5:3}, it is apparent that while the presence of the \isi{Progressive} is consistent with Non-stative interpretation, it is important to pay attention to the specific Non-stative interpretation that arises in each case rather than the mere fact that this is possible. In the case of \textit{run}, consistent with its status as a \isi{Process}, with \isi{Progressive} aspect it is interpreted as an ongoing \isi{Process}. In the case of \textit{close}, however, due to its status as Change of state\is{State!Change of} item (\isi{Transition}), what the use of the \isi{Progressive} captures is a Change of state\is{State!Change of} in progress. 

Furthermore, in the case of \textit{have} the \isi{Progressive} viewpoint allows for its interpretation as an ongoing \isi{Process} even though \textit{have} in and of itself may be labelled a \isi{Stative} predicate. In that case, the \isi{Progressive} aspect rather than the \isi{inherent aspect} of the verb may in this case be accredited with establishing a viewpoint where the verb \textit{have} in \textit{having a headache} is not interpreted as a \isi{State}. Rather it is interpreted as a \textit{series} of States -- somewhat analogous to an ongoing \isi{Process}. \citet{Guéron2008} addresses this influence of the \isi{Progressive} morpheme stating that, 

\begin{quote}
the ING morpheme which heads the participle in which the lexical vP is embedded ``massifies'' the spatial configuration vP denotes by multiplying its internal states. (p.~1822)
\end{quote}

The approach articulated by Guéron seeks to account for the interaction between different elements in the composition of \isi{Aspect}. Such an approach may be contrasted with the approach of \citet{Vendler1967} which claimed a restriction in the occurrence of \isi{Progressive} marking and verbs referring to States or Achievements. \citet{Lyons1977} addresses this incompatibility of ``stativity'' and ``progressivity'' as explicable in terms of the ``ontological distinction between static and dynamic situations'' (p.~707). Within an analysis which takes into consideration the contribution of the different elements involved in \isi{Aspect}, \textit{have} as used here may be analysed as a \isi{State}, a [\textminus Change] verb interacting with the \isi{Progressive} aspect -- an interaction which results in an aspectual viewpoint that may be classified as Non-stative [+Change]. Despite  the different interpretations associated with the use of \textit{have} in \REF{ex:5:1} above, it may be analysed as a verb that is inherently associated with the value [\textminus Change].

We will see for CECs that the presence of the \isi{Progressive} (\isi{Imperfective}) has more significance than just the indication of Non-stativity. In particular, I will argue in the case of items which appear in \isi{dual aspectual use} that while some of these are inherently Non-stative and thus compatible with Non-stative meaning, others are morphologically derived to express Non-stativity. Of those that are derived, we will see that some are derived to express a \isi{Process} and others a Change of state\is{State!Change of}. The difference is not just based on the ability of these items to appear in Non-stative use but rather the subtle differences that are evident in their Non-stative interpretations. 

\subsection{Non-stative use: Transitive alternation}\label{sec:5.1.2}
\isi{Transitive} alternation will be used here as a complement to the \isi{progressive criterion} as a means of evaluating interpretations that arise in the Non-stative use of dual aspectual items. This alternation as a method for the evaluation of Non-stative meaning will only be applicable to forms which appear to vary with regard to whether or not they express an \isi{Agent} or Cause. In \chapref{ch:2} we saw that along with the interaction of items with \isi{Progressive} aspect, work on \isi{dual aspectual forms} in CECs also provided data where the introduction of an \isi{Agent} or Cause allowed for Non-stative interpretation (see \sectref{sec:jaganauth} in particular).

The verb \textit{close} is a canonical example of an item that allows for this alternation in English as shown below (recall also the discussion of the causative\slash inchoative alternation in \sectref{sec:4.4.1}):

\ea\label{ex:5:4}
 \citep[53]{Pustejovsky1991}
\ea The door is \textbf{closed.}
\ex The door \textbf{closed.}
\ex John \textbf{closed} the door.
\z
\z

In these examples the item \textit{closed} points to a \isi{State} in (\ref{ex:5:4}a) with no reference to a Change of state\is{State!Change of}; a Cause or \isi{Agent} is not expressed or even implied. In (\ref{ex:5:4}b) the situation expressed is one which captures a Change of state\is{State!Change of} with no indication of an \isi{Agent} or Cause although one may be assumed. This illustrates the \isi{Inchoative} use of \textit{close}. However, (\ref{ex:5:4}c) is a clearly agentive usage where \textit{John} is responsible for the Change of state\is{State!Change of} resulting in the \isi{State} of the door being closed. This illustrates the \isi{Causative} use of \textit{close}. 

This alternation between the transitive and intransitive uses of a given form is particularly relevant for forms which may not be compatible with \isi{Progressive} aspect but permit an alternation of this kind. Note for example that \textit{sik} `sick' in JC\il{Jamaican Creole} is only marginally acceptable in the \isi{Imperfective} but is fully acceptable in transitive use. Compare:

\ea%5
\judgewidth{??}
\label{ex:5:5}
\ea[??]{
\gll Di pikni \textbf{a} \textbf{sik}.\\
\textsc{art} child \textsc{asp} sick \\
\glt `The child is getting sick.'
}
 
\ex[]{
\gll Di fuud \textbf{{sik}} di pikni.\\
 \textsc{art} food sick \textsc{art} child \\
\glt `The food sickened the child\slash caused the child to be ill.'}
\z
\z

In similar cases where \isi{Imperfective} (\isi{Progressive}) aspect may not be immediately acceptable, the causative\slash inchoative alternation allows for an evaluation of the Non-stative interpretation. 

% % % \subsubsection{Summary}\label{sec:5.1.1.1}

Both the compatibility of an item with \isi{Imperfective} aspect and its participation in the causative\slash inchoative alternation will be accepted in this work as tests as it regards the appearance of an item in Non-stative use. Additionally, the type of interpretation that arises once Non-stative expression is allowed will be considered. As noted in the case of \textit{have} and \textit{close}, there is a difference in interpretation of these items in Non-stative use which cannot be attributed to the different ways in which they appear in Non-stative use (i.e.: \isi{Progressive} aspect and \isi{transitive variation}). I will propose that the interpretation of an item in Non-stative use provides an indication of the \isi{inherent event} type with which such an item is associated. 

In the following, I will look at the different interpretations that may be associated with CEC property items in Non-stative use. In my evaluation I propose a link between these interpretations and the \isi{inherent event} type with which they may be associated. It is based on this that I posit a possible categorisation of CEC property items in \sectref{sec:5.2}. 


\subsection{Event types and semantic interpretations: State, Transition and Process}\label{sec:5.1.3}\label{sec:5.1.3.1}

In this section I will elaborate what may be seen as semantic criteria in the classification of CEC property items. It will become apparent that while a number of these items appear in Non-stative use, only some of these may be analysed as inherently Non-stative. In particular, we will see that there is justification for a general categorisation of this group of items as inherent \isi{Transition} [+Change] as opposed to inherent \isi{State} [\textminus Change]. However, the identification of items which I evaluate as inherent States only \textit{appearing} in Non-stative use provides a rationale for a sub-categorisation among \isi{State} items yielding derived Transitions and derived Processes. I present the preliminaries of this analysis below.

% % \subsubsection{State, Transition and Process}\label{sec:5.1.3.1}

As it regards the three primitive event types with which a verb may be lexically associated, recall that these are \textsc{state, process}, and \textsc{transition} \citep{Pustejovsky1988,Pustejovsky1991}. A \isi{State} is ``an eventuality that is viewed or evaluated relative to no other event'' while a \isi{Transition} is a ``\isi{single eventuality} evaluated relative to another \isi{single eventuality}'' and a \isi{Process} is ``a sequence of identical eventualities'' \citep[22--23]{Pustejovsky1988}. These definitions when placed alongside the semantic behaviours observed for JC\il{Jamaican Creole} property items in \sectref{sec:3.3} will illuminate criteria for my analysis of this group of items as constituting inherent Transitions on one hand and inherent States on the other. States may be further subdivided to include those that may be derived as either Transitions or Processes.

The different interpretations associated with JC\il{Jamaican Creole} property items in Non-stative use as discussed in \sectref{sec:3.3} relative to the different event types with which an item may be associated form semantic criteria that I will discuss in this section. If we recall the discussion of property items in \sectref{sec:3.3}, we observed what appears on the surface to be three categories of items. Firstly those such \textit{ded} `dead' and also Colour items such \textit{red} `red' and `black' which in Non-stative use indicate a Change of state\is{State!Change of}. Secondly those of the type \textit{chupid} `stupid', \textit{saaf} `soft', \textit{haad} `hard' \textit{swiit} `sweet' etc. which do not appear in Non-stative use. The third group of items highlighted are those of the type \textit{jelas} `jealous' and \textit{bad} `bad' -- these are distinguished from others which appear in Non-stative use in that they do not indicate a Change from one state to another but rather an ongoing \isi{Process}. 

A preliminary evaluation of the \isi{semantic behaviour} displayed by property items relative to the three possible event types highlights two anomalies: The first is the case of items in Non-stative use which are consistent with \isi{Process}. Based on the fact that such items appear in \isi{Stative} use as well, it is reasonable to assume that they cannot be inherently associated with a \isi{Process} \isi{event type}. Taking as a premise that a \isi{State} is more basic than a \isi{Process} in terms of Event Structure (cf. \citealt{Pustejovsky1991}, also \citealt{Grimshaw1990}), the \isi{State} use must be taken as reflecting the inherent status of items of this type and the \isi{Process} use as derived. Also, it seems to be the case that the \isi{Process} use of these items is less frequent and somewhat marginal across Creoles. It is based on this observation that I will analyse items which express a \isi{Process} in their Non-stative use as inherently associated with the Event Structure \isi{State}, and derived to express a \isi{Process}. 

The second anomaly that I observe is the case of items of the type \textit{ded} `dead' relative to Colour items such as \textit{blak} `black' and \textit{red} `red' in Non-stative use. They appear similar on the surface in that in Non-stative use, they all indicate a Change from one state to another. However, as I will argue below in \sectref{sec:5.2}, there is a subtle distinction between these groups of items which allows for the categorisation of one as \textsc{inherent transitions} and the other as \textsc{derived transitions}. The distinction supplied is based on the notion of logical opposition. In my analysis, items of the type \textit{ded} `dead' in their Non-stative use are evaluated relative to a \isi{logical opposition of contrariety} (\DEAD:\ALIVE) based on their Event Structure. Items of the type expressing Colour in Non-stative use are also evaluated relative to a logical opposition; however I note in this use that they express a \isi{logical opposition of contradiction} as opposed to one of contrariety.\largerpage[-1] 

Where a \isi{logical opposition of contradiction} exists, the relationship is between x and its negative counterpart \textsc{not} x (i.e.: \textsc{black:not black}). In such cases, \citet{Horn1989}, observes that this kind of negation

\begin{quote}
cannot in general be read as opposition or contrariety: When we speak of the `not great' [...] we do not pick out `what is small' any more than `what is of middle size', rather we refer simply to what is different from the great. (p.~5)
\end{quote}

Based on this, I note a subtle distinction in the \isi{semantic behaviour} of property items which express a \isi{Transition}: Items of the type \textit{ded} `dead', \textit{raip} `ripe', \textit{sik} `sick', etc., in their Non-stative use, establish an opposition with their contraries whereby a Change of state\is{State!Change of} is logically linked to `alive', `green' (lit. `unripe'), and `well', respectively. In the case of items expressing Colour on the other hand, a Change of state\is{State!Change of} resulting in \textit{blak} `black' or \textit{red} `red' at best is linked to an opposition of \textsc{not black} or \textsc{not red} respectively, not to an opposition of contraries such as `white' or `green' for example. I assess this lack of specific information as associated with the fact that this Change of state\is{State!Change of} meaning is not one that is inherently part of the Event Structure of such items but derived. 

Based on these preliminary observations I will argue for a classification of CEC property items into Transitions and States. The class of \isi{State} is further subdivided to account for derived Transitions and Processes. The basic identifying features involved in the classification are shown in \REF{ex:5:6}: 

\ea%6
\label{ex:5:6}
Criteria for the classification of CEC property items
\ea ability to appear in Non-stative use (\isi{Progressive}\slash transitive)
\ex expression of a Change of state\is{State!Change of} interpretation (in Non-stative use)
\ex type of logical opposition (contradiction vs. contrariety) in Non-stative use
\ex expression of a Processual (\isi{Activity}) interpretation
\z
\z 

\noindent Below in \sectref{sec:5.2}, I will apply these criteria to a classification of property items in JC\il{Jamaican Creole}.\largerpage[-3]


\section{Property items in JC: Towards a classification}\label{sec:5.2}

The distribution of the syntactic and semantic features in \REF{ex:5:6} will identify an item as a \isi{State}, \isi{Transition}, or derived \isi{Transition} or \isi{Process}. As the feature table (\tabref{extab:5:7}) shows, an item associated with a pure \isi{State} Event Structure only is identified as lacking any of these features given the fact that it does not appear in Non-stative use. A \isi{Transition} is positive for all features indicated and is identified through its association with the expression of an opposition of contrariety. A derived \isi{Transition} is identified based on its expression of a \isi{logical opposition of contradiction} in Non-stative use. A derived \isi{Process} lacks the feature of a logical opposition and that of a Change of state\is{State!Change of} (\tabref{extab:5:7}).

\begin{sidewaystable}
\caption{A model for the classification of property items in CECs}
\label{extab:5:7}
\small
\begin{tabularx}{\textwidth}{QQQQQQ}
\lsptoprule

 Categories & { Appear in Non-stative use}

 (progressive\slash \isi{transitive variation}) & { Allow for \CHANGEOFSTATE interpretation} & Encode \textsc{processual interpretation} & {\ENCODE} 

 {{\textsc{logical opposition} of Contrariety}} & {\ENCODE} 

 \textsc{logical opposition} of Contradiction\\
 \midrule 
 {{\textsc{1.} \CHANGEOFSTATE (\isi{Transition})}}\footnote{There is a seeming absence of an account of the fact that items of this type appear also in \isi{Stative} use. Given the general presumption that all property items appear in \isi{Stative} use, my focus is on the question of their Non-stative appearances and their \isi{inherent aspectual status}.}&  \ding{51}&  \ding{51}&  \ding{51}&  \ding{51}& \ding{55}\\

\tablevspace

 \textsc{2a.} \STATE & \ding{55} & \ding{55} & \ding{55} & \ding{55} & \ding{55}\\
 
\tablevspace

\textsc{2b. state\slash derived}

 {{\CHANGEOFSTATE (\isi{Transition})}} &  \ding{51}&  \ding{51}&  \ding{51}& \ding{55} &  \ding{51}\\

\tablevspace
\textsc{2c.} \STATE\slash\textsc{derived process} &  \ding{51}& \ding{55} &  \ding{51}& \ding{55} & \ding{55}\\ 
\lspbottomrule
\end{tabularx}
\end{sidewaystable}


This model in \tabref{extab:5:7} can be applied to all CECs to yield a descriptive classification of property items from a universal perspective of event types and their semantic connotations. However, the classification for specific lexical items may differ dependent on the lexico-semantic conceptualisation associated with such an item in a particular language or dialect as would be evidenced by its syntactic and \isi{semantic behaviour}. In the sections below, I will apply the model above to JC\il{Jamaican Creole} for a specific categorisation of property items.


\subsection{Transitions in JC}\label{sec:5.2.1}

In \tabref{extab:5:7}, inherent Transitions are distinguished from all others, including derived Transitions based on the fact that they entail a \isi{logical opposition of contrariety}. This type of opposition is taken to be inherent in that it provides explicit information on the original state involved previous to the Change of state\is{State!Change of} in question. Consider the use of JC\il{Jamaican Creole} \textit{raip} `ripe' as shown below: 

\ea%8
\label{ex:5:8}
\ea
\gll Di plantin ra\textbf{ip}.\\
\textsc{art} plantain ripe \\
 \ea `The plantain is ripe.’ 
 \ex `The plantain ripened.'
\z

\ex 
 \gll Di plantin a ra\textbf{ip}.\\
\textsc{art} plantain \textsc{asp} raip \\
 \glt { `The plantain is becoming ripe\slash ripening.'}

\ex 
 \gll Di igla dem raip di mango dem fi sel.\\
\textsc{art} vendor {\textsc{pl}} raip \textsc{art} mango {\textsc{pl}} to sell \\
 \glt {`The vendors ripen the mangoes to sell (them)}.'
\z
\z

Note here that \textit{raip} `ripe' appears in \isi{Stative} use (\ref{ex:5:8}ai) and in clearly Non-stative uses in (\ref{ex:5:8}b) and (\ref{ex:5:8}c) which express in each case an overt Change of state\is{State!Change of}. The Non-stative interpretation that is supplied in (\ref{ex:5:8}aii), shows as well that such an item in \isi{Stative} use may be ambiguous between a ``regular'' \isi{State} reading and a resultative reading. This as we will see, seems not to be the case for any Class 2 States. 

What is characteristic of an item such as \textit{raip} `ripe' in JC\il{Jamaican Creole} is that in both its \isi{Stative} and Non-stative interpretations, it may be said to be interpreted relative to a logical opposition which provides explicit information on its original state which is contrary to the \isi{State} `ripe' In this case, in order for a Change of state\is{State!Change of} to \RIPE to take place; the understanding is that the original state of the item in question was logically \GREEN (or `unripe'). Thus, even for (\ref{ex:5:8}a) which expresses a \isi{State}; this current \isi{State} results from a processual Change of state\is{State!Change of} from \GREEN. In (\ref{ex:5:8}b--c), which express an explicit Change of state\is{State!Change of}, the same is true, this results in a \isi{State}. The Change of state\is{State!Change of} is logically linked to an original \isi{State} that starts out with the \GREEN and the Change of state\is{State!Change of} leads logically to a state of `ripeness' The difference in this case, however, is that the focus is on the \isi{Process} which results in the \isi{State} of `ripeness' rather than the \isi{State} itself (which has not yet been achieved). 

There are a number of items in JC\il{Jamaican Creole} which may be shown to display behaviours similar to that of \textit{raip} `ripe'. Some of these are \textit{kuul} `cool', \textit{hat} `hot', \textit{sik} `sick', \textit{wet} `wet', etc. These all allow for \isi{Progressive} aspect, transitive use and a Change of state\is{State!Change of} interpretation which entails a \isi{logical opposition of contrariety}:\largerpage

\ea%9
 \label{ex:5:9}
\ea 
\gll Di parij \textbf{{kuul/ hat}}.\\
 the porridge cool/hot\\
\ea `{{The porridge is cool\slash hot.'}}\\
\ex `{The porridge has cooled\slash has heated up}.' \\
\z

\ex 
\gll Di parij a \textbf{{kuul/hat}}.\\
 the porridge \textsc{asp} cool/hot\\
\glt `{The porridge is cooling\slash heating (up).'}

\ex 
\gll Im \textbf{{a kuul/hat}} di parij. \\
 \textsc{3sg} \textsc{asp} cool the porridge\\
\glt { `He\slash She is cooling\slash heating the porridge.'}
 \z
\z

\ea
\label{ex:5:10}
% (10)
\ea 
\gll Di kluoz dem \textbf{wet/jrai}.\\
 the clothes {\textsc{pl}} wet/dry\\
\ea {`The clothes are wet\slash dry.'}
\ex {`The clothes have become wet\slash dry.'}
\z
\ex 
\gll Di kluoz dem \textbf{a} \textbf{wet/jrai} pan di lain.\\
 the clothes {\textsc{pl}} \textsc{asp} wet/dry on the line\\
\glt {`The clothes are getting wet\slash drying on the line.'}

\ex 
\gll Rien \textbf{wet} di kluoz dem pan di lain.\\
 rain wet the clothes \textsc{pl} on the line\\
\glt `{Rain wet the clothes on the line}.'

\ex 
\gll Son \textbf{jrai} di kluoz dem pan di lain.\\
 sun dry the clothes \textsc{pl} on the line\\
\glt {`Sun dried the clothes on the line.'}
\z
\z


\ea%11
\judgewidth{??}
\label{ex:5:11}
\ea[]{
\gll Di pikni \textbf{sik}.\\
 the child sick\\
\ea `{The child is ill.'}
\ex `{The child became ill.'}
\z}

\ex[??]{ 
\gll Di pikni \textbf{a sik}.{\footnotemark}\\
 the child \textsc{asp} ill\\
\glt `{The child is getting ill.'}
}
\footnotetext{This expression is marginally acceptable at best in JC\il{Jamaican Creole}. Hence, the compatibility of lexical items with \isi{Progressive} aspect is not in and of itself a reliable test for (Non)-\isi{Stativity} of an item.} 

\ex[]{
\gll Di fuud \textbf{sik} di pikni.\\
 the food sick the child\\
\glt `The food made the child sick.'}
\z
\z

As shown here, lexical items such as \textit{kuul} `cool', \textit{hat} `hot', \textit{wet} `wet', \textit{sik} `sick', etc. like \textit{raip} `ripe' appear in both \isi{Stative} and Non-stative use. In the latter, they express a Change of state\is{State!Change of} within a logical opposition between \textsc{hot:cold}, \textsc{wet:dry}, and \textsc{sick:well}, respectively. 

Items of this type are those that are perhaps most suitably called \isi{dual aspectual forms}: They may be said to inherently allow for both the \isi{Stative} and Non-stative expressions as seen in the examples above. Nonetheless, items displaying such behaviour, although they appear in \isi{Stative} and Non-stative use, are best seen as inherently Non-stative and associated with the abstract semantic feature [+Change]. This is based on the \isi{event type} with which they are associated. Recall that the structure in \figref{ex:5:12} captures the \isi{event type} of \isi{Transition} that is expressed by items of the type shown in examples \xxref{ex:5:8}{ex:5:11}. 

% % \protectedex{\ea%12
\begin{figure}
\caption{\isi{Transition} Event Structure \citep[56]{Pustejovsky1991}\label{ex:5:12}}
% %  \begin{center}
\fbox{\parbox{5cm}{\centering
\begin{forest}
[{e$_0$\\{\footnotesize ~~~[transition]}} 
 [{e$_1$\\{[\isi{Process}]}}]
 [{e$_2$\\{[\isi{State}]}}] 
]
\end{forest}
}}
\end{figure}


Recall that a \isi{Transition} is the merger of the notions \isi{Process} and \isi{State} where each is taken to be evaluated relative to the other. In other words, a lexical item having the Event Structure of \isi{Transition}, when expressing a \isi{State}, is understood to entail a \isi{Process}; when expressing a \isi{Process}, a \isi{State} is assumed as the endpoint. 

Thus, upon a \isi{Stative} interpretation (e\textsubscript{2}), an item displaying the range of behaviour of JC\il{Jamaican Creole} \textit{raip} `ripe’, \textit{sik} `sick’, \textit{kuul} `cool’, \textit{hat} `hot’, \textit{wet} `wet’, etc. must be understood relative to an opposition. Moreover, it must be understood in relation to the \isi{Process} (e\textsubscript{1}) that accounts for the culmination of the relevant \isi{State}, although, in such a case, the (semantic) focus is on the \isi{State} (e\textsubscript{2}), i.e.: an attribute (result) rather than any \isi{Process} that brought it about. This I believe distinguishes this kind of \isi{State} interpretation from forms which are truly \isi{Stative}, in that this \isi{Stativity} is a resultative part of a larger template for a single word that is Non-stative whereas purely \isi{Stative} forms do not share this same complex structure. While it may be argued that even in the case of \isi{Stative} verbs there must have been a point at which that \isi{State} was entered in, this is usually not a part of the meaning of the word itself and does not form a part of its Event Structure.

In previous studies in CECs, authors have been accustomed to analysing verbs purely in terms of an opposition between \isi{Stativity} and Non-stativity (cf. \citealt{Bickerton1975,Jaganauth1987,Winford1993,Gooden2008}; etc). Based on Pustejovsky’s identification of three distinct event types at this level, the basic opposition \isi{Stative}\slash Non-stative may be applied to the Event Structure of \isi{State} and \isi{Process} respectively. However, the third Event Structure of \isi{Transition} adds a complexity to this opposition in that it merges both concepts (\isi{State} and Non-state) within one \isi{event type}. This ``merger'' allows conceptually for two primitive event types determined by whether the focus is on the \isi{Process} or on the resulting \isi{State} within a \isi{Transition}: What may be called a \textsc{processual transition} which focuses on a \isi{Process} within the \isi{Transition} and the resulting \textsc{transitional state} which focuses on the \isi{State} within the \isi{Transition}. The structure in \figref{ex:5:13} is intended to capture this.

% % \ea%13
 \begin{figure}
 \caption{The two foci of the \isi{Transition} Event Structure \label{ex:5:13}}
\fbox{\parbox{5cm}{\centering
\begin{forest}
[{e$_0$\\{\footnotesize ~~~[transition]}} 
 [{e$_1$\\\scshape processual\\\scshape transition}]
 [{e$_2$\\\scshape transitional\\\scshape state}] 
]
\end{forest}
}}
\end{figure}

\figref{ex:5:13} captures the shift in focus that a \isi{Transition} Event Structure allows, where a speaker may choose to highlight the \isi{Stative} or Non-stative aspect of a situation. This type of choice is somewhat analogous to the choice that a speaker has as it regards \isi{viewpoint aspect} where for example a verb indicating a \isi{State} may be used in the context of \isi{Progressive} viewpoint resulting in an overall Non-stative outlook (cf. \citealt{Smith1983}). 

The representation in \figref{ex:5:13} reflects the fact that the \isi{Stative} meaning (e\textsubscript{2}) expressed within the context of a \isi{Transition} is distinct from that of a pure \isi{State} in that it entails a Change of state\is{State!Change of}. Likewise the Non-stative meaning conceptually entails a resultant \isi{State} and in this way is distinct from the \isi{Process} associated with a \isi{Process} Event Structure. Based on this, what we observe in the different uses of this class of \isi{dual aspectual forms} is a shift in focus between a Processual \isi{Transition} in the case of the Non-stative interpretation and a Transitional \isi{State} in the case of the \isi{Stative} interpretation. Both these are linked to a \isi{Transition} Event Structure that is inherently Non-stative or [+Change]. 


\subsection{States among JC property items}\label{sec:5.2.2}
% % % \subsubsection{ ``Pure'' States}\label{sec:5.2.2.1}
The second class of items based on the proposed model, are those which are States. This general group is diverse in that it contains items which may also be modified to express Non-stativity (\isi{Transition} or \isi{Process}). The most obvious items that would fit this class, however, are those which do not allow for Non-stative interpretation, the ``pure'' \isi{State} items. This means that they are not compatible with \isi{Progressive} aspect and do not participate in the \isi{transitive alternation}. Examples of such items are shown below: 

\ea%14
\label{ex:5:14}

\ea[]{ 
\gll Di siment \textbf{haad}.\\
 The cement hard\\
\glt `The cement is hard.'\\
 {* `The cement has hardened.'}}
 

\ex[*]{
\gll Di siment a \textbf{haad}.\\
 the cement \textsc{asp} hard\\
\glt {`The cement is hardening.'}
}

\ex[*]{
 \gll Dem \textbf{haad} di siment.\\
\textsc{3pl} hard the cement\\
\glt `{They made the cement hard.'}
}
\z
\z

\ea%15
 \label{ex:5:15}
\ea[]{
\gll Di lemanied \textbf{swiit}.\\
the lemonade sweet\\
\glt {`The lemonade is sweet.'\\}
* `{The lemonade has been sweetened.}'}

\ex[*]{
\gll Di lemanied a \textbf{swiit}.\\
 the lemonade \textsc{asp} sweet\\
\glt {`The lemonade is getting sweet.'}
}

\ex[*]{
\gll Dem \textbf{swiit} di lemanied.\\
 \textsc{3sg} sweet the lemonade\\
\glt {`They sweetened the lemonade\slash made the lemonade sweet.'}
}
 \z
\z 


\ea%16
\label{ex:5:16}
\ea[]{Da man de \textbf{chupid}.\\
 that man \textsc{loc} stupid\\
\glt `That man is stupid.'\\
 * `That man has become stupid.'}

\ex[*]{
\gll Da man de a \textbf{chupid}.\\
 that man \textsc{loc} \textsc{asp} stupid\\
\glt {`That man is getting\slash behaving stupid.'}
}

\ex[*]{
\gll Di uman chupid di man.\\
 the woman stupid the man\\
\glt `The woman made the man stupid.'
}
\z
\z

\ea%17
 \label{ex:5:17}
\ea[]{
 \gll Di riva \textbf{waid/lang/braad},\\
\textsc{art} river wide/long/broad\\
\glt `The river is wide\slash long/broad,'\\
 `The river has been widened\slash lengthened/broadened.'}

\ex[]{
\gll di riva a \textbf{waid/lang/braad},\\
\textsc{art} river \textsc{asp} wide/long/broad\\
 \glt `The river is widening\slash lengthening/broadening.'
}

\ex[*]{
 \gll Dem a \textbf{waid/lang/braad} di riva.\\
\textsc{3pl} \textsc{asp} wide/long/broad \textsc{art} river\\
\glt `They are widening\slash lengthening/broadening\slash the river.'}
\z
 \z

As shown here, there is a clear group of items in JC\il{Jamaican Creole} which are restricted in their ability to appear in Non-stative use. Examples such as \textit{haad} `hard', \textit{swiit} `sweet', \textit{chupid} `stupid', \textit{waid} `wide', \textit{lang} `long' and \textit{braad} `broad' are used in the examples above to show this restriction.

Items which display this type of behaviour are classified here as \isi{State} consistent with the Event Structure in \figref{ex:5:18}.

% % \ea%18
\begin{figure}\caption{\isi{State} Event Structure \citep[56]{Pustejovsky1991}\label{ex:5:18}}
\fbox{\parbox{3cm}{\centering
\begin{forest} [S [e]] \end{forest}}}
\end{figure}
% % \z  
 
Recall that the structure in \figref{ex:5:18} represents an \isi{event type} that is not viewed or evaluated relative to any other event. What this means essentially is that the item behaves the way it does because of how it is conceived by the speaker or the community. Thus it may be said that in JC\il{Jamaican Creole}, items such as \textit{haad} `hard', \textit{swiit} `sweet', \textit{chupid} `stupid', \textit{waid}  `wide’, \textit{lang} `long' and \textit{braad} `broad' are not conceived as inherently involving a Change of state\is{State!Change of} and are not open to the introduction of an external Cause or \isi{Agent}. 

In reality however, it must be noted that there may be a degree of flexibility in the behaviour of an item across Creoles, or across dialects within a Creole and even among individual speakers. Thus the thrust should be towards a behavioural model which accounts for the fact that an item is able to behave the way it does. A case in point is the behaviour of \textit{braad} ‘broad’. The cognate item \textit{bradi} in \ili{Sranan} appears in both \isi{Stative} and Non-stative use as seen in \sectref{sec:3.3}. Likewise, there is a possible classification for JC\il{Jamaican Creole} that would include \textit{braad} in a different category to that indicated above. Consider \REF{ex:5:19} for example\largerpage

\ea%19
 \label{ex:5:19}
 JC\il{Jamaican Creole}\\
     \gll Mi        a       waak chuu           di   duor wid           di bag eng   dong pan        mi shuolda    an          mi beli   \textbf{a} \textbf{braad}.\\
 \textsc{1sg} \textsc{asp} walk through \textsc{art} door with \textsc{art} bag hang  down on \textsc{1sg} shoulder and \textsc{1sg} belly  \textsc{asp} broad\\
\glt `I am walking through the door with the bag hanging on my shoulder and my belly broadening out (looking broad).'
 \z

This sentence was produced\footnote{February 23, 2010.} by a JC\il{Jamaican Creole} speaker from St.\ Elizabeth within the following context: She arrives at work and, entering through the door which has a mirror, she catches a glimpse of herself in the mirror. What I listed as a \isi{property item} with an Event Structure of (pure) \isi{State} – an Event Structure which does not allow for Non-stative expression, in this case appears in Non-stative use. The meaning indicated here suggests that at least for this speaker, JC\il{Jamaican Creole} \textit{braad} `broad' may be conceived as a \isi{Transition} or a \isi{State} of the type that is open to the introduction of an external Cause or \isi{Agent}, in other words a \isi{State} which may be derived as a \isi{Transition} or a \isi{Process}. The distinction here would depend on whether the perspective given is one where the bag was responsible for the `broadening' of her belly (Cause) or whether her belly just appeared broad as she walked. This is nonetheless an interesting use of the \isi{Progressive} aspect to express a viewpoint and would deserve further examination in a precise classification for this variety of JC\il{Jamaican Creole}.


The point made here is that any actual classification of particular lexical items must be flexible enough to allow for the variation that is evident in this group of items. Also, a model in this regard should have the tools to account for why it is that an item is able to behave the way it does. In this instance, it appears that while a lexical item such as \textit{braad} in one variety of JC\il{Jamaican Creole} may be a \isi{State} and not open to the introduction of Non-stative elements of meanings, in another variety, it may allow for derivation into a \isi{Transition}. Ultimately, a classification of an item must take as its point of departure its specific behaviour within the context of the variety under study in relation to the syntactic and semantic criteria outlined in \tabref{extab:5:7}. 

In the sections below, I will look at \isi{State} items in Non-stative use. 

\subsection{On the Non-stative use of State items}\label{sec:5.2.3}
\subsubsection{Derived Transitions}\label{sec:5.2.3.1}

A closer look at the group of items which may be classified as States shows that there are also those which, in contrast to items \xxref{ex:5:14}{ex:5:17} allow for Non-stative use. Consider first items denoting Colour such as \textit{red} `red' and \textit{blak} `black': 

\ea%20
 \label{ex:5:20} 
\judgewidth{??}
\ea[]{
\gll Di shuuz \textbf{blak}.\\
\textsc{art} shoes black\\
\glt  `The shoe is black.'}

\ex[??]{
\gll Di shuuz a \textbf{blak/red}.\\
\textsc{art} shoes \textsc{asp} black\\
\glt `The shoe is getting black.'
}

\ex[??]{ 
\gll Dem \textbf{blak/red} di shuuz.\\
\textsc{3pl} black/red \textsc{art} shoes\\
\glt `They are making the shoe black\slash blackening the shoe.'}
\z
\z

 Items expressing Colour such as \textit{red} `red' and `black' as shown here, in Non-stative use would only be marginally accepted in JC\il{Jamaican Creole}. However, there are cases where these items may be shown to be acceptable in Non-\isi{Stative} use. Consider the examples \xxref{ex:5:21}{ex:5:23} for example:


\ea%21
 \label{ex:5:21}
\gll im \textbf{red} im ‘an dem wid jragan blood\\
\textsc{3sg} red \textsc{3sg} hand {\textsc{pl}} with dragon blood\\
\glt ‘He used dragon blood\footnote{Note that `dragon blood' is a plant whose leaves when rubbed together produce a red substance that may be used as a kind of dye.} to redden his hand.'
 \z

\ea%22
 \label{ex:5:22}
\ea 
\gll Di mango dem \textbf{red}.\\
\textsc{art} mango {\textsc{pl}} red\\
\glt {} `The mangoes are red.'

\ex 
\gll Di mango dem \textbf{red} pan di chrii.\\
\textsc{art} mango {\textsc{pl}} red on \textsc{art} tree\\
\glt `The mangoes got red on the tree.'

\ex 
\gll Di son \textbf{red} di mango dem pan di chrii.\\
\textsc{art} sun red \textsc{art} mango {\textsc{pl}} on \textsc{art} tree\\
\glt `The sun reddened the mango.'
 \z
\z

\ea%23
 \label{ex:5:23}
\ea 
\gll Di doti gyas \textbf{blak} di pat dem.\\
\textsc{art} dirty gas black \textsc{art} pot {\textsc{pl}}\\
\glt `{The dirty gas made the pot black/sooty.'

\ex 
\gll Chuu di gyas doti di pat dem \textbf{a} \textbf{blak}.\\
 because \textsc{art} gas dirty \textsc{art} pot {\textsc{pl}} \textsc{asp} black\\
\glt `The pots are getting black\slash sooty because of the dirty gas.'} (that is used for cooking)
 \z
\z

In the examples, \xxref{ex:5:20}{ex:5:23} the items \textit{red} `red' and \textit{blak} `black' in Non-stative use, express what I analyse as the Event structure of a derived \isi{Transition}. Similar to the semantic interpretation associated with Transitions, in \REF{ex:5:20}, the marginal Non-stative use of \textit{red} `red' and \textit{blak} `black', denotes a Change of state\is{State!Change of}. I treat this as derived based on the fact that they do not in their Non-stative use establish the same type of logical opposition as inherent Transitions which I associate with an opposition of contraries. Thus, where JC\il{Jamaican Creole} \RIPE:\textsc{unripe} or \textsc{mad:sane}, express a logical opposition where a particular state is implied as the original state of the item, in the Non-stative use of an item such as \textit{blak} `black', this is not so. Essentially, there is no implication about the initial state of the item; in \REF{ex:5:20} for example, the \textit{shuuz} may have been white, purple, blue or faded black etc. The introduction of elements of meanings such as \BECOME and \CAUSE allow for the Change of state\is{State!Change of} interpretation that is apparent in (\ref{ex:5:20}b--c) but these are not linked in an opposition of contrariety. 

The marginal Non-stative use of these items as shown in \REF{ex:5:20} would not be enough to provide a basis for a class of derived Transitions among JC\il{Jamaican Creole} property items. However, the acceptability of the Non-stative use of these colour terms in \xxref{ex:5:21}{ex:5:23} indicates a need for an account that extends beyond the \isi{Stative} appearance of these items to account for the fact that they may also appear in Non-stative use. The notion of a derived Event Structure is an attempt to capture and account for such a usage. Note however, that there are differences in the meanings that are expressed in the Non-stative use of these colour terms. In \REF{ex:5:21}, the use of the item \textit{red} `red’ is clearly related to the colour term, but \REF{ex:5:22} and \REF{ex:5:23} may be analysed as idiomatic uses (i.e.: not as regular Colour denoting term). In (\ref{ex:5:22}a--b) it is used to refer to the Change of state\is{State!Change of} which results from the sun causing the mangoes to appear \textit{red} `red’ through burning thereby establishing an opposition between `burnt' and `not burnt'. Similarly \textit{blak} `black' in \REF{ex:5:23} is used to express a Change of state\is{State!Change of} resulting in `sooty'. 

The idiomatic uses these items seem to suggest the possibility of lexical items different from the colour terms themselves and thus a different categorization altogether, possibly that of inherent \isi{Transition}. The behaviour of these items in this regard would merit further investigation. Nevertheless, I would like to point out even in these idiomatic uses, what appears to be a clear association with the Colour terms \textit{red} `red' and \textit{blak} `black': The burning of the mango physically results in the colour \textit{red.} Likewise, although `sooty' represents a special kind of \textit{black} that is arrived at through burning and smoke, the physical result is the colour black. Note as well that the usage of \textit{red} `red' in \REF{ex:5:21} reflects one that is not idiomatic. 

Based on these observations, I am inclined to maintain a categorisation of these as derived Transitions within the group of States among property items in JC\il{Jamaican Creole}. Beyond JC\il{Jamaican Creole} as well, work by \citet{Alleyne1987} shows items of this type (expressing colour) occurring in Non-stative use for \ili{Sranan} suggesting that such a Class is relevant to CECs as well. 

The examples in \xxref{ex:5:20}{ex:5:23} seem to establish a contrast with those items shown in \xxref{ex:5:14}{ex:5:17}. However, despite the seeming separation between such forms based on syntactic criteria, I would like to suggest that these are unified as inherently associated with an Event Structure of \isi{State}. In order to account for the ability of the items in \xxref{ex:5:20}{ex:5:23} to appear in Non-stative use, I posit that these are open to a \isi{morphological process} which introduces Non-stative elements of meanings into their Event Structure. This is consistent with the representation of \citet{Carter1976} (discussed in \chapref{sec:4}) which shows a relationship between \DARK:\DARKEN. 

In analysing the Non-stative use of such items I would like to propose that their Event Structure is one that is similar to that of a \isi{Transition}. However, as shown in \figref{ex:5:24}, the Change of state\is{State!Change of} aspect is derived through the addition of another Event Structure level.

\begin{figure}
% % \ea
\caption{An Event Structure representing derived \isi{Transition}\label{ex:5:24}}
 \fbox{\parbox{5cm}{\centering
\begin{forest}
[derived transition, s sep=1cm
 [e$_1$]
 [S,tikz={\node [draw,inner sep=0,fit to=tree,circle]{};}
 [e,]
 ]
] 
\end{forest}}}
\end{figure}
% % \z

Note here that a \isi{State} (S) Event Structure as represented by \citet{Pustejovsky1991} forms the base of this structure. This is merged with the Event Structure representation of \isi{Transition} to show the introduction of Non-stative elements of meanings, namely those that are associated with a first sub-event (e\textsubscript{1}) that would result in a \isi{State}. What I propose is that the Non-stative version of an item such as \textit{blak} `black' arises through a covert \isi{morphological process} that affects the Event Structure of the \isi{State} item \textit{blak} `black' allowing for the expression of a \isi{Transition}. In this way \textit{blak} `black' is able to express a Change from one \isi{State} to another; however, this Change of state\is{State!Change of} initiated by e\textsubscript{1} is not inherently linked to the logical contrary of \textsc{black} which is \textsc{white}. 

Two Non-stative possibilities are evident based on the examples in \REF{ex:5:20}: The inchoative version (\ref{ex:5:20}b) which shows the \textit{shuuz}  `shoe(s)’ as the affected entity (Theme) in subject position and the transitive version which shows a Cause or \isi{Agent} in the position of subject (\ref{ex:5:20}c). In the case of the inchoative version, it may be posited that a semantic primitive \BECOME is covertly introduced into the representation of the lexical item that accounts for this expression. The primitive \BECOME is directly related to the Theme argument of the \isi{State}, thus there is no change in the syntactic structure. 

However in the case of the transitive version (\ref{ex:5:20}c), it is a \CAUSE primitive that may be said to account for the change in the syntactic structure, namely the introduction of a Cause\slash \isi{Agent} and the transitivity that contrasts with the \isi{Stative} version in (\ref{ex:5:20}a). Regarding this, consistent with the implications of the structure in \figref{ex:5:24}, \citet{Grimshaw1990} observes that a Cause argument will be associated with the ``first sub-event which is causally related to the second sub-event'' (p.~26). Thus, the introduction of the primitive \CAUSE accounts for the change in transitivity that is seen in (\ref{ex:5:20}c) as opposed to (\ref{ex:5:20}a).

In \sectref{sec:5.2.3.2}, I will look at the case of derived \isi{Process} as they appear in JC\il{Jamaican Creole}.

\subsubsection{Derived Processes}\label{sec:5.2.3.2}

There is a second group of items among States which, like the \isi{State} items discussed above appears in Non-stative use. However, in Non-stative use such items are consistent with a \isi{Process} Event Structure. Examples of these are shown in \REF{ex:5:25} and \REF{ex:5:26}: 

\ea%25
 \label{ex:5:25}
\ea 
\gll Dat de pikni \textbf{bad/ruud}!\\
that \textsc{foc} child bad/rude\\
\glt `That child is a bad\slash rude child!'

\ex 
\gll Dat de pikni a \textbf{bad/ruud} lang taim.\\
 that \textsc{foc} child \textsc{asp} bad/rude long time\\
\glt `That child has been misbehaving for a long time.'
 \z
\z

\ea%26
\label{ex:5:26}
\ea
\gll Dem \textbf{jelas}.\\
 \textsc{3pl} jealous \\
\glt `They are jealous.'

\ex 
\gll Dem a \textbf{jelas} mi fi di kyar we mi jraiv.\\ 
\textsc{3pl} \textsc{asp} jealous \textsc{1sg} for \textsc{art} car that \textsc{1sg} drive\\
\glt `They are (being) jealous\slash envious of me because of my car.'
\z
\z

In these examples, items such as \textit{bad} `bad', \textit{ruud} `rude' and \textit{jelas} `jealous', which appear in \isi{Stative} use, also appear in Non-stative use. In contrast to those items denoting colour, they provide an interpretation that is consistent with a \isi{Process}. The difference between these and standard Processes such as \textit{run, push}, etc. is that the latter do not appear in \isi{Stative} use. 

I analyse items displaying this behaviour as inherent States derived to express a \isi{Process} Event Structure. This is based on the primacy of \isi{State} \citep{Grimshaw1990} and the fact that inherent Processes do not also express States. In these derived Processual uses, I associate such items with an Event Structure of \isi{Process} as shown in \figref{ex:5:27}.

% % \ea%27
\begin{figure} 
\caption{Event Structure of \isi{Process} (\citealt[56]{Pustejovsky1991})\label{ex:5:27}}
\fbox{\parbox{5cm}{\centering
\begin{forest}
[P
 [e$_1$,no edge]
 [~~...~~,roof]
 [e$_n$, no edge]
]
\end{forest}
}}\end{figure}
% % \z 

In contrast to the ``eventuality that is viewed or evaluated relative to no other event" in the (a) examples, the examples in (b) are more consistent with a ``sequence of identical eventualities" \citep[22]{Pustejovsky1988}. 


\section{A classification of property items in JC}\label{sec:5.3}

% \ea
\begin{sidewaystable}
\small
\caption{A classification for property items in JC\label{extab:5:28}}
\begin{tabularx}{\textwidth}{lQCCCCc}
\lsptoprule
Categories & JC\il{Jamaican Creole} items & \multirow{2}{\hsize}{Appear in Non-stative use\footnote{\isi{Progressive}/transitive variation}} & \multirow{2}{\hsize}{Allow for \CHANGEOFSTATE\footnote{(\textsc{non-stative}) interpretation}} & \multirow{2}{\hsize}{Encode \textsc{processual} interpretation} & \multicolumn{2}{c}{Encode a logical opposition}\\\cmidrule(lr){6-7}
           &          &                           &                          &                                           & \textsc{contrariety} & \textsc{contradiction}\\
                                                                            \\\midrule
\textsc{1.} \CHANGEOFSTATE\footnote{Transition} & \itshape mad, wet, kuul, raip, sik, hat ded drai, ful sik & \ding{51} & \ding{51} & \ding{51} & \ding{51} & \ding{55} \\
\textsc{2}a. \STATE                             & \itshape saaf, haad, chupid wotlis, sowa nyuu, oul, big, fain, fat, maawga, braad, etc. & \ding{55}& \ding{55}& \ding{55}& \ding{55}& \ding{55}\\
\textsc{2}b. \textsc{derived change}  & \itshape blak, red & \ding{51} & \ding{51} & \ding{51} & \ding{55} & \ding{51}\\
\hspaceThis{\textsc{2}b.}~\textsc{of }\STATE\footnote{Transition}\\
\textsc{2}c. \textsc{derived} \isi{Activity} & \itshape jelas, bad, ruud, etc. & \ding{51} & \ding{55} & \ding{51} & \ding{55} & \ding{55}\\
\lspbottomrule
\end{tabularx}
\end{sidewaystable}
% \todo[inline]{add classification}
% \z

Based on the examination of JC\il{Jamaican Creole}, the classification in \tabref{extab:5:28} may be posited. The classification in \tabref{extab:5:28} represents a behaviour-based model which classifies a lexical item in accordance with its semantic and syntactic behaviour. Note that this categorisation that I posit may vary from that which may be observed for other authors dependent on the variety of JC\il{Jamaican Creole} that is under study. For example, \citet{Bailey1966} points to the occurrence of an item such as \textit{fat} with \isi{Progressive} aspect (cf.: \textit{Im a fat} `S\slash He’s getting fat', p. 47). What this indicates is variation in the categorisation of fat dependent on the variety of JC\il{Jamaican Creole} under study. Similarly, in the case of \textit{braad} `broad' as we saw in \REF{ex:5:19}, the different behaviours associated with a particular item may warrant a different categorisation based on the specific variety under study. 

The model in \tabref{extab:5:28} is designed to accommodate the variability that has been observed in the behaviour of this group of items. In this regard, it may be taken to replace that of \citet{Winford1993}, which as discussed in \chapref{ch:3}, fails to capture the variability in behaviour that exists across semantic categories. Note for example here that the group of Transitions for JC\il{Jamaican Creole} includes items from the semantic categories of Human Propensity (\textit{mad}), along with those classified as expressing Physical Property. Similarly, the \isi{State} items are a mix of those expressing Physical Property, Dimension and Age. In this way, the actual behaviour of an item is captured in its classification. 

The analysis that I propose here, although based on earlier semantic-based works such as that of \citet{Carter1976,McCawley1968,Dowty1979,Pustejovsky1988,Pustejovsky1991} and \citet{Grimshaw1990}, is also compatible with syntax-based approach\-es such as that articulated in \citet{Larson1988}, and \citet{Travis2010}. In particular, regarding Larson’s VP shell analysis, it may be noted that a VP shell is associated with each primitive element that is included in the proposed analysis. This accounts, for example, for the fact that; the introduction of \CAUSE at the level of primitive Event Structure, translates to a corresponding Cause argument at the level of vP. Similarly the introduction of \DO at the level of Event Structure means the introduction of an \isi{Agent} argument at the syntactic level of vP. This is consistent with the distinct differences observed between clearly \isi{Stative} and Non-stative versions of dual aspectual items. 

\section{Summary}\label{sec:5.4}

What we have seen in this analysis of \isi{dual aspectual forms} may be summarised as a general descriptive approach to this category of items in CECs captured in the following questions:

\ea%29
\label{ex:5:29}
 \ea {\justifying Does a particular item allow for both \isi{Stative} and Non-stative interpretation? (All Class 1 items do, class 2 is divided)}
 \ex {\justifying In the Non-stative variation, is there an inherent opposition between contraries i.e.: is specific information implied on the initial state of the item in question? This separates items that I have labelled as Class 1 items from those Class 2 items that may be morphologically derived into Change of state\is{State!Change of} predicates (2b).}
 \ex {\justifying Is the Non-stative interpretation one that includes Change through Agency but does not include a Change of state\is{State!Change of}? This separates items of Class 2b from 2c.}
 \z
\z 

The treatment that I propose here for \isi{dual aspectual forms} addresses the class of property items in CECs as diverse, based on their semantic and syntactic behaviours. From the perspective of an Event Structure analysis, items within this general group are evaluated and classified consistent with the notions of \isi{Transition}, \isi{Process} and \isi{State}. In this way, they are associated with a unique aspectual value in accordance with their Event Structure. The notion of a \isi{Transition} Event Structure and the introduction of meaning components consistent with the expression of Change accounts for the duality in aspectual expression that we see in those items which appear in both \isi{Stative} and Non-stative use. This treatment effectively separates inherently \isi{dual aspectual forms} from other items which are either purely Processual or \isi{Stative} in character. 

This analysis is not restricted to a particular (Creole) language but may be applied to a treatment of property items generally. Since it is expected that different language communities will have different conceptualisations associated with particular items, the actual categorisation of items may differ across speech communities. Nevertheless, general behavioural patterns consistent with event types will be observed allowing in each case for language specific generalisations over sets of items. In the following chapter I will analyse the implications that such an analysis has for the \isi{categorial status} of property items.
